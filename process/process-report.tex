% procesverslag, voor zover relevant
% voor het begrijpen waarom het eindproduct
% in deze context de beste oplossing is:
% stappen met de beslissingen en de argumenten daarbij,
% plusminus 2 pagina's.
% Wellicht is dit geen getrouwe geschiedschrijving,
% maar een manier om de opdrachtgever en de examinator te laten begrijpen
% waarom dit een goed product is gegeven de kaders en doelstellingen

To structure or own \gls{source-code}
we looked into using a so called ``seed project''.
This is a default template code base and file and folder structure
to get started quickly and relieving you from developing your own architecture.
We found the project
Ultimate-Seed\footnote{\url{https://github.com/pilwon/ultimate-seed}}
which looked promising.
But after a closer examination we concluded it was outdated
and would require to much effort to get to know the full workings,
update and use for the \gls{project}.
So we decided to start from scratch,
concerning the architecture of our own code base.
This way we know exactly what we have,
and we only add code and dependencies when we really need them.
Our code base will be kept clean, small and organized.
And everything will be documented for future teams to get started easily.
