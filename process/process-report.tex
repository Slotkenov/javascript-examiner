% procesverslag, voor zover relevant
% voor het begrijpen waarom het eindproduct
% in deze context de beste oplossing is:
% stappen met de beslissingen en de argumenten daarbij,
% plusminus 2 pagina's.
% Wellicht is dit geen getrouwe geschiedschrijving,
% maar een manier om de opdrachtgever en de examinator te laten begrijpen
% waarom dit een goed product is gegeven de kaders en doelstellingen
%
% + Modular approach for the checks
% + Backend Node.js or other language
% + Backend Seed project or from scratch
% + Frontend Polymer, AngularJS or from scratch/Twitter Bootstrap
% - Unit testing Jasmine or Mocha/Chai

This chapter will be a discussion about the choices we made
during the process of developing the \gls{examiner}.
The choices regarding modeling and development decisions will be discussed.
The decision made will be pointed out,
including arguments as to why we thought
that to be the best decision for this product.

\section{Checks}
The \gls{examiner} consits of various \glspl{check} for examining \gls{js-code}.
We built a couple of these \glspl{check}.
Other teams will pobably continue this project after us
and will add \glspl{check} to the \gls{examiner}.
A modular approach for the \glspl{check} seemed mandatory to enable
subsequent teams in working on the project.
As explained in the Architecture chapter,
the \glspl{check} have direct access to the database
and the frontend has direct access to the \glspl{check} via HTTP.
There is no coupling with other backend logic.
The frontend coupling is one way
and via HTTP a clean interface is provided.
Only the direct coupling with the database could be discussed.
The \glspl{check} have complete access to all data in the database
and could read and write anything they want.
Another option would be to provide a RESTful interface ---
like we have for the frontend ---
which serves as the interface for the \glspl{check} to communicate with the database.
We choose the approach of direct access to the database
to not be too restrictive on the \glspl{check} on one hand,
and to save time for other requirements on the other hand.
And while making this choice it is not set in stone,
it would be easy to implement an interface
between the \glspl{check} and the database later on if deemed necessary.

\section{Frontend}
At the very beginning we started out
just writing some standalone HTML and \gls{js} for our frontend needs.
But soon we saw the need for a more structured way at the frontend as well,
especially since we wanted to develop a \gls{spa}.
We looked at a few different possibilities.
Namely: AngularJS or Polymer.
To address the styling of our application we could use Twitter Bootstrap.

AngularJS is a fairly new framework which makes creating templates a breeze.
It can be combined with Twitter Bootstrap for a good looking app.
The downside of AngularJS is that when your needs grow more specific
and you need to create your own custom elements
(called directives in AngularJS),
AngularJS tends to get very complicated.

Polymer is even newer than AngularJS
and has been discussed at length in the Frontend chapter.
While not used much and not documented very well yet,
Polymer has a great way of creating custom HTML elements
making use of web standards.
Therefor it is future proof
and becomes more stable automatically as browsers implement these standards.
Besides that it comes with a set of custom elements
complete with styling,
which makes it not necessary to bring in another styling package
like Twitter Bootstrap.

Nobody of us had experience with Polymer
and we thought this a good opportunity to get to know it.
Polymer seems to be framework with a lot of potential.
With documentation that could be better
and not much to find about it on the Internet
it is a bit tough to start with.
But that might pay of very well in a few years
when Polymer has gained some traction.
It remains a guess and you always have to find a balance
between safety and innovation,
but in the still so young, and quickly shifting area of web development
you have to think ahead a few steps all the time.

\section{Backend}
When determining what backend server and language we would use
we quickly found out about
Node.js\footnote{\url{https://nodejs.org/}}.
This seemed to be a good approach for the prototypical nature of the project.
And, while not necessary, nicely integrates with the \gls{examiner}
because \gls{js} is used to write code in Node.js.
You can easily set something up in Node.js in minutes,
there are modules available for sheer anything
--- we found some modules which could be of great use to us ---
and it is no problem to create a lightweight webserver in Node.js as well.
At the very start of the project we also briefly thought of using
Haskell\footnote{\url{https://www.haskell.org/}}.
Being a functional programming language
it might lend itself well for complex analysis of \gls{js-code}.
However, none of us had (much) experience with Haskell.
To be able to write code in the same language we were examining,
and with all of the other advantages of using Node.js
we did not see the need
to dive deeper into other possible options for a backend.

While it is so easy to start working in Node.js
you can get lost quickly in the determining a good structure
for your code and files,
and which module you should pick for a certain task.
To remedy this we looked into using a so called ``seed project''.
This is a default template code base and file / folder structure which enbables
to get started quickly and relieving you from developing your own structure.
We found the project
Ultimate-Seed\footnote{\url{https://github.com/pilwon/ultimate-seed}}
which looked promising at first.
Unfortunately, after a closer examination we concluded it was outdated
and would require to much effort to grasp the full workings,
update the seed project and use it for the \gls{project}.
As a result we decided to start from scratch,
developing our own structural and architectural code base.
This way there is no overhead, as the dependencies were added when actualy 
needed. Hence, our code base will be kept clean, small and organized.
Furthermore it was easy to keep everything documented for the ease of future 
teams.

\section{Maintainability}
To keep our code maintainable, it is good to write unit tests.
There are actually two great modules for running and writing unit tests:
Jasmine and Mocha (combined with Chai).
Jasmine is an all in one package with a testrunner,
logic for creating a test suite and the possibility to create spies.
Mocha is a testrunner only
and you want to combine it with Chai for writing your testsuite
and maybe you need to include Sinon as well for writing spies.
It does not really matter that much if you choose for an all in one package
or use the separate packages; you include them once, and you're done.
We chose to use Mocha with Chai over using Jasmine.
Jasmine is a great package for creating and running unit tests for the frontend
but has not much focus on the backend.
Mocha and Chai however are targeted at both frontend and backend.
This fact and the fact that \gls{wac} uses Mocha and Chai as well
was enough for us to decide to use Mocha and Chai
for writing an running unit tests for the \gls{examiner}.
