% teamreflectie op het project en de gekozen aanpak:
% wie heeft welke rollen vervuld,
% hoe verliep het project,
% plusminus 2 pagina's
After being assigned to this project, we started to create a project plan.
The distribution of tasks related to this phase went quite well. As a result,
we figured it would be no problem proceeding without assigning distinct
roles. The tasks should be rotating and the roles would shift accordingly.
This approach went quite alright to the extent we all actively participated in
the planning of the project and the communication with the product owner and 
mentor. Of course, the dynamics in the team led to some accents, where Ronald
often showed his accommodating nature, Boris turned out to have the most extensive
knowledge of available tools and the way to implement them and Bram utilized
his managing experience. This naturally formed separation of roles, led to a
situation wherein Ronald was willing to self assign the practical tasks like 
communicating progress with the tutor and arranging approval for our project 
plan. Boris used his experience of the technical field
to discern important success indicators concerning the possibilities of 
automatic feedback and the technical implementation,
whereas Bram shifted towards the questions related to the purpose of the project
and the topic of feedback in general. 

After writing the project plan, we worked separately on our individual Domain
Research paper. For Boris and Bram, the topic of their papers derived quite 
naturally from their orientation. From this perspective, the subject
of Ronald's initial paper was somewhat more artificial, especially
when considering it as a scientific oriented assignment. After finishing the 
papers, Ronalds effort resulted in an artifact that supports the urge for the
current revision of the ABI course. In our opinion, the tension
between the practical background (both of the course as of the study) and 
scientific focus (of this course) was brought to light here. 

During this period of the separated duty (apart from reviewing each others 
work), the roles formed in the first phase were diminished, probably as they 
were not defined. It took a while before the 
collaboration was recovered, and we only reached a really effective state after
we started to use the issue management features from
GitHub\footnote{\url{https://github.com/blog/831-issues-2-0-the-next-generation}}.
This way we gained more insight of progress and were able to effectively divide
tasks. 

At this first phase, we decided to go along with Rational Unified Process
development. This choice was rather pragmatic then the outcome of thorough
research: we all studied the course Object Oriented Analyses and Design, which
handles a great deal of the RUP artifacts. Moreover, the phases seemed to 
correlate with the structure of the course. Only when we arrived at the 
implementation phase, the duality of the course became clear. We more or less
asserted the first two phases of the course could be mapped to the first two
phases of RUP. This turned out to be an omission. The Elaboration phase was far
from concluded. Luckily we recognized this almost immediately, and were able
to transform the first implementation oriented iteration to a mainly 
elaborative iteration. Thanks to stepping back, we managed to postpone some
fundamental questions (like building from scratch or using an existing
framework) and perform some research. It was great to experience the
flexibility in the team, as the acceptance of taking a step back was
embraced immediately. 

During the construction phase, the characteristics mentioned before reinforced
quite clearly. Boris demonstrated his superior implementation skills, 
while Ronald and Bram tried to keep up in trying to master the used platforms. 
Whereas Ronald hesitated to push his code to the repository, Bram was quite 
assertive in adding his sometimes a bit hackisch code. Luckily, we were watching 
each others contributions. This was not only an end, but necessary because of
the way we separated our tasks. In example, after Boris finished the core client
side functionality, Bram added the exercise- and user management, extending the
% I don't understand the sentence below, don't know what to make of it,
% please revise (Boris)
code written by Boris, reviewing this code as necessary part of the process
to being enabled adding new functionality.

Furthermore, we discussed problems we stumbled upon, analyzing each others code.
Because we met at Brams place every other week, it was literally possible to
watch over each others shoulder to overcome problems. Besides this approachable
pair-programming method, the live meetings were quite fruitful, in particular
for discussion. Besides it can be very motivating to actually physically work
together when working together on the same project. 

At the end of each implementation iteration, we presented the results to the product
owner. As we lacked clear separated duties, it was not clear who was responsible 
for the demonstrations. This resulted in a somewhat messy presentation. Despite
our inconvenience in this matter, we did not manage to improve. Before 
the last demo we noticed the urge to improve, but we failed to act
accordingly. Our aspiration to add as much functionality as possible combined
with complications with merging the code hindered a tight, well dressed
presentation. On the upside, the magnitude of our final presentation won't be
underestimated. Bram has been assigned to create a blueprint as a starting
point.

In retrospective, it might have been fruitful to identify the several roles
(chairman, secretary, technical leader), as well as recognizing the course as
both a practical project and a scientific exploration, instead of a fluent
combination. On the other hand, without a clear division we were still able --- or maybe
even just because of it --- to adjust on shifting insights and expectations, 
work together in harmony, learning quite a lot from each others qualities and
last but not least to develop the \gls{examiner}.
