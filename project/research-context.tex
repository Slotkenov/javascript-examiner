\documentclass{article} 

\usepackage[utf8]{inputenc}
\usepackage{hyperref}
\usepackage[square, numbers]{natbib}

\begin{document} 

\title{Research Context}
\author{Ronald Kluft \and Bram Nieuwenhuize \and Boris Arkenaar}
\date{\today\\v1.0}
\maketitle 

\section{Introduction}
%Research group: IDEAS
%Tool: Ask-Elle
%Focus: research question 2 and 3: easily adding and finetuning exercises by a tutor 
%Corresponding requirements:
%	- The tutor should be able to create an exercise (...)
%	- Code Layout definition
%	- Programming guidelines definition

%Questions
%	-Is it possible to get insight in the actual process of exercise creation?
%	-How many exercises have been created by tutors? 
%	-How is their reception of the exercise creation process? (research already in progress?)
%		* What are the difficulties they experience?
%		* Are their didactic aspects they can't, but would like to introduce? 
%		* What features are used most? (i.e. Global vs Location specific)
%	-What were the arguments to work with model solutions? Have other possibilities
%	 been considered? (i.e. test suites)
%	-Is there a limit for the magnitude of an exericse? What are implications (towards possibilities, scope etc.)?
%	-Would it be possible to create a set of exercises in a way there is no need for additional material?
%	-Are there indicators for a tutor on which constructs feedback annotations are needed? How does
%	  the retrieve these indicators?
%	-Is there a size limit or minimum for a  provided set of model solutions?
%	-The correctness is based on equality. Is the current measure method to determine (in)equality rightly balanced?
%	-Is it possible to get a demo on actually adding a simple exercise?
%	-Is Ask-Elle flexible in usage? (mainly in the way exercises are added)
%	-How to proceed if a there are 3 model solutions, and two solutions
%	  need feedback on the same particular construct, should this
% 	  feedack annotation be global?
%	-How does Ask-Elle make a distinction between the different solutions (from optimal to suboptimal)?
%	@TODO: meer specifieke vragen (met name hoofdstuk 7 + hoofstuk 5 (feedback scripts))

\bibliographystyle{abbrvnat}
\bibliography{bibliography}

\end{document}