
\documentclass{article} 
\usepackage{hyperref}
\begin{document} 

\title{Basic feedback on JavaScript code} \author{Bram Nieuwenhuize (851514132)} 
\date{\today} \maketitle 

\section{Introduction} This article contains the outcome of a limited study 
towards 
some aspects of providing feedback on JavaScript code. The foundation of this 
research is a project which goal is the development of a software tool called 
JavaScript-examiner
\footnote{\url{https://github.com/Slotkenov/javascript-examiner/}}.
The aim of this tool is to help students to learn 
JavaScript in an academic distance learning environment, in particular
while studying the course Webapplications: The Clientside 
\footnote{\url{http://tinyurl.com/lph6k5s}}. This research is one of 
the artefacts to realize some contex and generate content, to extract 
requirements and recommendations for the remainder of the main project. \newline 
From more or less general feedback as starting point, quite soon the scope is 
narrowed to the specific field of interest: Providing automatic feedback on 
JavaScript code. At first, a theoretical foundation is outlined, to create an 
overview. Some thoughts are given to Distance Learning, E-learning and 
learning a (program) language.
This overview is used to penetrate the rich field of (freely 
available) existing online platforms to learn JavaScript, in enquiring the
presence of didactics as proposed in academic writings and discovering some 
characteristic features. In addition, a tutor with experience in examining
the mentioned course is 
interviewed to get more insight in difficulties and commonly made mistakes while 
programming in JavaScript. The results of these two field explorations are 
comprehended, to determine success factors of such a platform as is on the one 
hand and, more importantly, elucidate requirements and features for the 
JavaScript-examiner, that will contribute to mastering the art of programming in 
JavaScript. 

\section{Feedback in learning a programming language}
When developing software it is often quite important to get familiar with the 
environment where the software will be used. In this case, developing for
students like ourselves (even within the same discipline), it might be argued
this knowledge is already present. To some extend this is true, like our
experience in learning programming language and even learning them aided by 
software designed for this purpose. At the same time, there is a lack of
knowledge and expertise in teaching, providing feedback and 
motivation techniques. In assuming these aspects are important for the success
of the project, as supported in the conclusion of this article
\footnote{Investigating online learning 
environments in a web-based math course in Jordan,
Akram M. Alomari, Yarmouk University, Jordan, International Journal of Education
and Development using Information and Communication Technology
(IJEDICT), 2009, Vol. 5, Issue 3, pp. 19-36}
 , some related articles are consulted.

\subsection{Distance Education \& E-learning}

Next to the support of students, another goal of the project is to reduce the 
amount of time spent by the tutor to correct the (in JavaScript written) 
solutions to exercises. According to Guri-Rosenblit
\footnote{‘Distance education’ and 
‘e-learning’: Not the same thing
SARAH GURI-ROSENBLIT
Department of Education and Psychology, The Open University of Israel, P.O. Box
39328, Ramat-Aviv, 61392, Israel} it is hard to achieve this goal by using 
e-learning. This study focuses on the E-learning and distance education, 
showing
examples where e-learning is not helpful in this aspect, and even 
increases the workload. This is something to keep in mind, as the bias of 
ICT automatically leading to a decrease of labour-intensity is often quite strong.
Later on, some potential preventive measures to this concern
are presented. \newline Another interesting article 
\footnote{G.S. Ypsilandis (2002) Feedback in Distance Education, Computer
Assisted Language Learning, 15:2, 167-181, DOI: 
\{url{10.1076/call.15.2.167.8191}} about feedback and distance education 
outlines some interesting observations. A shift from teaching towards learning
is recognized in contemporary education. Feedback is more and more 
recognized as important assistance mechanism. This feedback should not only
take place from tutor to student, but also among students themselves and even
the student providing feedback to the self. According to (
mentioned in the article) Ross(2000), there are 
three characteristics of feedback quality: timely, helpful, developmental. It
should be provided at the right time within the learning process, have some 
instrumental aspects to really help the student to reach a next level and it 
should contribute in motivating to continue studying. Besides these general 
key issues, the field study points out two specific requirements. It should be
possible to provide feedback on the exercises and the platform, in a way the
student experiences some kind of dialogue. The other
recommendation is to provide feedback in a print friendly format, as a lot of 
students still prefer reading from paper when studying, and store
the feedback with the other
in hard copy provided materials. This last point is connected to the conclusion 
of Shepherd
\footnote{Graduate student preference for instructor feedback in MBA
distance education,
Morgan Shepherd,
University of Colorado at Colorado Spring}, namely the demand of well written
(full sentences) 
feedback instead of just some key points or audiovisual indicators.

\subsection{Similar Projects}
Now. with some general context about the field of use and feedback in general,
it's time to move on and
start to narrow the scope. To start a project like the JavaScript-examiner can
be hard, if there is no guidance. Luckily there have been similar approaches. 
One of these approaches is a project with a similar goal, with the significant
difference being the programming language, that is Java instead of JavaScript. 
This project \footnote{Java Tutor: 
Bootstrapping with Python to Learn Java,
Casey O’Brien, Max Goldman, Robert C. Miller,
MIT CSAIL,
Cambridge, MA 02138 USA
cmobrien, maxg, rcm, @mit.edu} has an interesting approach, for using a 
supposedly known language (Python) to support the learning process of the 
language to be learned (Java). %Argumenten waarom dit een goede methode is...
Besides the didactic aspects of this method, it's a way to reduce the,
earlier mentioned, labor intensity for the tutor. Creating exercises by 
putting the solution more or less down in another well known programming 
language is likely
to be far less challenging and time saving, compared to creating a
a set of delimited solution requirements and problem description
in well written natural language. The goal to reduce time of the tutor is even
stretched further in this project. Students are supposed to write test cases 
themselves. The Tutor only writes a very basic test suite. Students are 
challenged to 
add test cases to this suite. It's even demanded to enable the feedback
tool itself. This seems to be a little too much, as it undermines the primary
goal, and the skill of writing such test cases requires knowledge of the
language in the first place. Still, the idea of rewarding the student for an
contribution like adding a test case is something to keep in mind. To avoid a 
lot of similar test cases, a Code Coverage Library is used. Thus in short, 
the tool in the mentioned project needs a problem statement written in a known
program language, a template for the language to learn, a (at first basic) 
test suite. \newline
Another project 
\footnote{Learning Programming Languages
through Corrective Feedback and Concept Visualisation,
Christopher Watson1, Frederick W.B. Li1, and Rynson W.H. Lau2,
1 School of Engineering and Computing Sciences, University of Durham, 
United Kingdom
2 Department of Computer Science, City University of Hong Kong, Hong Kong
christopher.watson,frederick.li,@durham.ac.uk,
rynson@cs.cityu.edu.hk} 
focusses on the form of the offered exercises. Their concern is the 
drawback of little and  thoroughly delineated assignments. From a
maintainability and developmental perspective these might be in favour 
to complex and realistic quests. According to their research, there is a demand
for interesting and realistic problems to solve. This challenges the student to
really grasp the fundamental skills of programming, and motivates the student as
they experience there is really something to gain, in solving such a puzzle.
In order to motivate students, their goal is creating a model to enable 
development of adaptive e-learning systems to provide individualized guidance 
and feedback. From a presupposed link between learning and playing, the model
is inspired by the concept of Serious Games: games with educative purpose. 
According to them, games are appropriate because of their nature in developing
skills through repetition and extending knowledge step by step. But, 
compared to most serious games, more comprehensive feedback is 
required, due to the complexity of learning a programming language. The proposed
model is quite comprehensive. In creating learning scenes (like levels in games)
, provided with a set of learning concepts, pre-required scenes (to create 
gradual complexity), a test suite and code similarity analyse (functional 
solution) and
required fragments and constraints (semantic solution). In comparison with the 
project above, the amount and complexity of artefacts needed for an example is
much greater. But in creating more complex exercises, the amount of exercises 
needed to cover the material might be reduced to some extend. The model includes 
an approach for step by step feedback as well. This feature might be interesting 
for later stages or probably even subsequent projects to extend the 
JavaScript-examine. In relation to this last point, this article
\footnote{It’s Alive! 
Continuous Feedback in UI Programming,
Sebastian Burckhardt Manuel F¨ahndrich,
Peli de Halleux Sean McDirmid,
Michal Moskal Nikolai Tillmann,
Microsoft Research,
fsburckha,maf,jhalleux,smcdirm,micmo,nikolaitg@microsoft.com,
Jun Kato
University of Tokyo,
i@junkato.jp}
might be interesting as well.

\section{Existing Online Javascript education platforms} 
Possessed with the background information, now it is possible to
review some current platforms that offer learning programming languages like 
JavaScript. Though, before jumping in, it is important to have some plan to
review the platforms. As determined in last chapter, the following 
characteristics and features will be reviewed:
\begin{itemize}
  \item Platform (web-based, command shell, client, user interface)
  \item Exercises
    \begin{itemize}
      \item Form(i.e. natural language, programming language, game)
	  \item Extend
	  \item Structure(i.e. gradual difficulty)
	  \item Challenging?
    \end{itemize}
  \item Input format (i.e. template, empty text area, command-line)
  \item Feedback
    \begin{itemize}
	  \item General characteristics
	    \begin{itemize}
          \item Timely
	      \item Helpful
	      \item Developmental
        \end{itemize}
	  \item Form (i.e. Well written, bullet points, model solution, print friendly)
	  \item Extend (i.e on syntax, on functionality, on semantics)
	  \item Contributors (i.e. Creators of platform, users)
	\end{itemize}
  \item Possibilities to send feedback to the platform and exercises
\end{itemize}

Of course, any not yet distinct, but possible interesting features, 
will be examined as well. As the goal is to get an overview of the aspects and
facets rather then the specifics of each platform, the review is ordered by 
the distinct facets. Subsequently, the specific platforms are only
explicitly mentioned if
a present feature is considered useful to examine more closely later on.

\subsection{Platform}
%Web-based, command shell, client, user interface
All platforms are web based, except Nodeschool.IO
\footnote{\url{http://nodeschool.io/}}
, which uses the Node.JS 
Command Line and NPM. One platform is visually presented as a game
\footnote{\url{http://codecombat.com/play}}. Most platforms offer learning
modules for several languages. Some platforms use cookies, 
others require registration and login to keep track of progress. Login is 
sometimes required to start learning. Some of the platforms enable logging in 
with Google or Facebook credentials. \newline
In general, the user interface consists of a big area to write and / or show 
code, a quite small area with instructions, an area to show output and some 
indication of location and direction (the place of the current exercise within 
the course). On top of this, KhanAcademy
\footnote{
\url{https://www.khanacademy.org/computing/computer-programming/programming/}} 
show responses and related discussions about the current exercise as well. Most
platforms require a mouse click to submit the solution, and some clicks again to
move on to the next exercise. In codeschool
\footnote{\url{http://javascript-roadtrip.codeschool.com/}} 
it's possible to navigate through exercises by using keyboard short-cuts or 
keywords.

\subsection{Exercises}
%Form, Extend, Structure, Challenging
All platforms have the exercises divided in several sections. The introduction
starts often with a classical challenge: print 'Hello World' to the console. 
KhanAcademy, Code Combat and Make Games With Us
\footnote{\url{http://www.makegameswith.us/build-flappy-bird-in-your-browser/}}
have chosen a different introduction, in providing some functions that should be
called to generate a visual action on the screen. For example, in Code Combat 
you have to collect Gems by moving the Hero through calling moveRight() on 
the this keyword. \newline
In some cases, there is a clear learning curve, and it's required to complete a
section to move on to the next one. Others, like Nodeschool.IO, just present 
different sections where distinct features within a section are linked to an
exercise. While examining the platform, I felt more attracted to the platforms
with a learning curve. But Nodeschool.IO enables the user to quickly learn 
something about a specific topic, without the need to complete any other
sections. A solution that have both benefits might be the possibility to 
complete a section at any time, but mention the pre-required knowlegde for each
section. \newline
The extend of covered programming aspects depends slightly, but the greatest 
part is still about core programming within a single module. Advanced software
engineering practices like modularity, design patterns, code elegance or 
performance indicators (i.e. use of memory, running time) are hardly covered. 
Exceptions are KhanAcademy (sections Object Oriented Design, Documentation, 
Readable code), and Nodeschool.IO (requiring for example recursive functions, 
restrictions on using specific constructions or functions). \newline
The exercises itself range from one liners like print something to console, to
(pseudo) realistic application like building or playing a little game. 
The scale of the latter is still quite small and is written within one module.
Often, a section starts with the smallest exercises, and finishes with a big 
one where all aspects of the section come together. These big exercises often 
are more challenging and really demands the user to make use of the newly 
acquired skills. Sometimes it is hard to remember required facets from earlier
sections, when demanded in a different one. Here, it would have been useful to 
make use of an (optional) refresh. Almost all platforms with a explicit
learning curve have divided the 
material in the following sections: Introduction, Strings, Logic / Loops, 
Arrays, Functions, Objects. 

\subsection{Input}
%Template, text area, command line
As already mentioned, NodeSchool.IO is somewhat different to the other platforms
, and this goes along for the Input as well. It's the only platform where any
text editor can be used. To provide some guidance, all exercises come with a 
template that can be copied to the newly created text file. All other platforms
use a rich text area, often already provided with a template consisting of 
declarations, methods and commented instructions. The text areas are rich in the
sense it's possible to undo or redo actions, cut, copy, select (multiple lines 
as well). Udacity
\footnote{\url{https://www.udacity.com/}} has also some simple input textfields
for the purpose of submitting answers in natural language. KhanAcademy has a
nice feature, where some information about the exercise is provided dynamically
in the same textarea where the user can create the solution. This works quite
nice, as there is no explicit switch from passive to active learning.

\subsection{Feedback on solutions}
%General: Timely, Helpful, Developmental
%Form
%Extend
%Contributors

\subsection{Feedback on platform and exercises}
KhanAcademy has a nice functionality in this aspect as well. All exercises are 
linked to a topic that is shown under the exercise. Within this topic, users are 
helping each other to solve the problems. Some other platforms have some kind
of forum or q \& a as well, accessible through a link that opens a new window. 
This shift of view is quite disorientating so the barrier to make use 
of this feature is quite high. The possibilities to provide feedback towards the 
platform itself is invariably limited, only a general contact function is 
provided. 

\section{General difficulties in learning JavaScript} 

\section{Conclusion} 

\end{document} 