When working on a project, it's important to become familiar with the affected
domain. In this case, the current project is potentially only a first step to
reach the eventual goals. This was kept in mind when selecting topics for the
domain researches. Instead of focusing only on relevant aspect for the current
project, the scope is quite broad. As a result, the research 
not only managed to get direction for our implementation of the application, 
but also discerned possible success indicators and functionality that stretches 
out to subsequent projects.

During the preliminary inquiry to the ends and aspects for the \gls{examiner},
three actors came to surface. The student, that will use the tool primarily to
improve his \gls{js} knowledge and skills. the tutor, who wants to reduce the
labor intensity of examination while maintaining the ability to track the
students progress (net yet implemented.
And the researcher, researching the learning process of \gls{js},
the best way to provide \gls{feedback}
and maybe even how to configure the application to improve the students'
performance. At first sight, the role of the researcher may seem the main
concern for this section. Sure, the research is by no means bound to any
limits that can be found when extracting the context wherein both student and tutor
act. After some consult with the stakeholder, it became apparent that the
extent to which research can be supported with this tool is subordinate to the
goal of successfully supporting the tutor and student. In a way, the opportunities
for the researcher depend heavily on the actual usage by the other actors.
In fact, great part of the validity of this project can only be measured when 
there are actual participating users. From this perspective, three distinct 
subjects where selected. 

Feedback is one of the main topics for this project. Being already familiar with 
some online tools that can provide \gls{feedback} based on code input, it seemed 
valuable to research on feedback and in particular how \gls{feedback} is 
presented currently, both in the academic environment (through an interview) as 
well in the available online tools. To learn from the relevant domain as well, 
this research focuses on \gls{feedback} related to \gls{code} or even \gls{js}. 
With this research we got an idea of feedback in relation to the domain, an 
overview of currently available similar tools and got a better notion of the 
process from a tutor perspective.

Another domain that can't be overlooked is the domain of the programming
language itself, and particular the quality indicators that can be derived. In
order to be able examining code, it is required to know what can be examined and
how this examination may be grouped. This research can immediately help to 
decide what should be examined, and can indicate the feasibility factors. In
addition, the research revealed more fundamental related research questions, which
led to the main topic of the research context consult.

The domain of the didactics is covered in the research on designing and developing
in a structured way. Following a recipe could lead to improved \gls{js-code}. In
conjunction with the, closely to this project related course \gls{wac}, a first
exploration on stimulating or enforcing the usage of such a recipe is elaborated.

These three researches not only reveal a better overview for the current project:
new questions are invoked, a project transcending scope is shaped and some practical
limitations came to surface. The research papers in the following sections are
written in a way they can be published on their own. As a result, the introductions
contain some redundant information when reading it as part of the main thesis.

To comply with the formal requirements of the ABI course, only the paper written
by the formal author (specified on the front page) of this exemplar is included 
as part of this thesis, in the following section. The remaining two papers are 
added as appendices \ref{app:recipe} and \ref{app:semantics}.
