\section{Introduction} 

Our goal is to make life easier for a tutor
of the \gls{js} programming language.
In our bachelor project we will develop a \gls{examiner}
which will relieve the \gls{tutor} from checking
all \glspl{student}' \glspl{solution} to \glspl{exercise}
and give them appropriate feedback in an online course.
At the same time, the \gls{examiner} will benefit the \glspl{student} as well.
The \gls{tutor} simply doesn't have time
to check the \glspl{solution} of all the \glspl{student}.
And therefore the \gls{tutor} cannot give individual \gls{feedback}
to each \gls{student}.
The \gls{tutor} will pick the biggest and most common pitfalls he sees
in the \glspl{solution} he picks.
Then he will give the \glspl{student} general feedback on those pitfalls
in the next online group session.

The reason why the \gls{examiner} will benefit \glspl{student} --- besides
relieving the \gls{tutor} of too much of a workload --- is threefold.
First off the \gls{examiner} is able
to check all the \glspl{exercise} of all \glspl{student}.
Secondly, the \gls{examiner} can give the \glspl{student}
immediate \gls{feedback} on their \gls{exercise} ---
instead of them having to wait for
the next online group session with the \gls{tutor}.
And finally the \gls{examiner} gives the \gls{student}
individual \gls{feedback}.
Instead of some general explanation
about common pitfalls the \gls{student} might not even have had trouble with,
the \gls{examiner} gives each \gls{student} \gls{feedback}
on the specific problems encountered in his or her \gls{solution}.

This article will be an analysis of the quality aspects of \gls{js-code}
on which the \gls{examiner} can focus.
What kind of quality aspects exist for programming languages in general?
To make sure the scope of this article won't get to wide
I will only pick a few quality aspects.
I will base my choice on what we have learned at the Open University
and on well documented quality aspects found in literature.

How easy is it to measure those quality aspects for a given \gls{code}?
and how precise is the \gls{feedback} of those measurements?
For measuring quality aspects
I will look at literature about available measurements,
how well they are documented
and how difficult it would be to implement such a measurement.
The results of the measurements of different quality aspects can vary from
a general idea about the quality aspect of a given \gls{code}
to more specific indication of how a \gls{code} might be improved.
I will compare the measurements found in this article
on the impact of their results.
More on how \gls{feedback} is given to the student
can be read in the domain research of
Bram Nieuwenhuize\footnote{domain-research-feedback.pdf}.

\section{Code quality}

\Gls{code-quality} can be divided into two areas.
In one area you look at the quality of the \gls{code} itself,
without executing it.
You look at aspects like how the \gls{code} is laid out,
how easy it is to read the \gls{code}
and how well you can understand what the \gls{code} is trying to do.
These aspects have no effect on the execution of the \gls{code} however.
The second area looks at the quality of execution. How well
does the \gls{code} execute? Does it use efficient \glspl{construct}?

We will start in this section by looking at the first area:
the quality of the \gls{code}.
First we will see how we can determine the quality of the \gls{code}
by looking at its layout.
Secondly we examine metrics for expressing the \gls{maintainability}
of a given \gls{code} and how we can use that
to give \gls{feedback} to \glspl{student}.

\subsection{Natural Language}

Although a programming language is an artificial language,
natural language is also used often in code in the form of
names (of variables, routines, etc.),
headers (documentation blocks above routines, classes, etc.)
and comments (explanation of the code between lines).
These are three important aspects for giving \gls{feedback}
as explained by \citet{stegeman2014empirically}.
Determining the quality of natural language is much more difficult
than for an artificial language.
Therefor it would be best not to start with this aspect
when development of the \gls{examiner} starts.
It is a very interesting subject though,
and it would be very valuable if the \gls{examiner} could give \gls{feedback}
on this area as well.
But this subject would need a lot of research and can be looked in to
in a later stage of the development process of the \gls{examiner},
probably for another team.

\subsection{Code layout}

A good \Gls{code-layout} can help you maintain an overview
of the various sections of a \gls{code}.
This is another important aspect for giving feedback
by \citet{stegeman2014empirically}.
It says something about the order in which routines are placed
and if routines are located close to the routines that call them.
Because this becomes an important aspect when your \gls{code} grows
it won't be important for us to implement into the \gls{examiner}.
We will be looking at small \glspl{exercise}.
When the \gls{examiner} will be expanded
and also be used for bigger \glspl{exercise}
it might become interesting to take a look at this aspect again.

\subsection{Code formatting}

For readability it is important to structure your \gls{code} in a way
that conveys clearly the intention of the \gls{code},
this is called \gls{code-formatting} \citep{stegeman2014empirically}.
There are many different ways you could format your \gls{code}
and there is not one formatting that is better than all the others.
Most important is consistency.

The easiest way to check for a consistent formatting would be
if one particular format would be required from the \glspl{student}.
Then the \gls{examiner} can simply check for that formatting.
Otherwise it would have to determine the formatting
of a particular \gls{exercise} on various aspects
and see if that formatting is used consistently
throughout the entire assignment.

Besides being consistent in using one formatting
there are still formatting aspects that would be considered bad practice
and should be avoided at all times.
There are many \glspl{tool} for examining \gls{code}
to make sure none of these bad practices are used in \gls{code}.
Even specific for \gls{js-code} such \glspl{tool} already
exist.\footnote{For instance JSLint (\url{http://www.jslint.com/})
and the less strict variant JSHint (\url{http://www.jshint.com/})}

\subsection{Maintainability}

Various metrics can be used to give an idea
about the \gls{maintainability} of \gls{code}.
We are not looking for measuring \gls{maintainability} specifically,
but an indication of it can help us determine the quality of the \gls{code},
for \gls{maintainability} is an aspect of good quality \gls{code}.
\citet{rakic2013problems} list the following metrics
used in tools for analyzing \gls{code}:
\begin{itemize}
  \item Lines of code;
  \item Cyclomatic complexity;
  \item Halstead complexity;
  \item Object oriented metrics.
\end{itemize}
We will look at the first three metrics mentioned.
They touch the subjects of flow, expressions, and decomposition
as described by \citet{stegeman2014empirically}
The last one --- Object oriented metrics ---
might also be an interesting metric
when object oriented aspects are introduced in the \glspl{exercise}.
But it would lead too far for this research to go into;
it can be the subject of a later research.

First off you can look at the amount of \gls{code}
in terms of lines of \gls{code}.
The more lines of \gls{code}, the more complex it gets.
There are different ways to determine the lines of \gls{code} though.
You can simply count all the lines in the source files,
which we will call lines of code (LOC).
But you could also skip the lines which contain comments and empty lines,
the remaining lines we will call source lines of code (SLOC).
Going one step further you can count the actual statements in the \gls{code}
instead of the physical lines.
That will give you the logical lines of code (LLOC).
A disadvantage of LLOC is that it is more difficult to calculate.
Instead of simply counting the lines of a text file.
You would need serious knowledge of the programming language in question.

Only looking at the lines of \gls{code} is not very useful.
We can compare the LOC of the \gls{student}
with the \gls{solution} of the \gls{tutor},
but that will give you only a rough idea about
whether the \gls{student} used far too much \gls{code} or extraordinarily few.

A more precise measurement of the complexity of code
might be cyclomatic complexity.
This basically tells you in how many unique paths
your \gls{code} could be executed \citep{website:js-complexity}.
When the cyclomatic complexity increases,
so does the complexity of your \gls{code}.
The more possible execution paths,
the more difficult it is to reason about your \gls{code}.
But because the cyclomatic complexity increases
when the code base increases
it is more useful to express the cyclomatic complexity relative to
the lines of \gls{code}.
This gives us the cyclomatic complexity density \citep{gill1991cyclomatic}.
Also important is to make sure that
the bodies of conditional statements don't get to large.
And routines should as well contain only a limited set of tasks and variables
\citep{stegeman2014empirically}.

The third metric we will be discussing is the Halstead complexity
which looks at operators and its operands.
It determines the amount of operators and operands in a given \gls{code}.
It also determines the amount of distinct operators and operands.
With that data some calculations are performed
to determine the complexity of the \gls{code}.

While it is not necessary to fully understand
the programming language for performing the Halstead metric,
that method has some shortcomings as explained by \citet{yu2010survey}.
Where the cyclomatic complexity looks at the control flow of the code,
the Halstead method only looks at the operations performed,
but ignores the control flow completely.
When used together however you can look at your \gls{code} from both sides
and express the \gls{maintainability} in a combination of the results.

\section{Execution quality}

The quality of a given \gls{code} can be determined by looking at
the \glspl{construct} that have been used,
and by looking at the manner in which the \glspl{construct} have been combined.

There is an interesting way to look at a new programming language
which can also help us here to look at \gls{js-code},
of what structures it is composed and how they are combined.
These words from Abelson and
Sussman\footnote{Abelson, H., Sussman, G.J. 1984
  {\em Structure and Interpretation of Computer Programs}.
  MIT Press, [\url{https://mitpress.mit.edu/sicp/}]}
give us that way of thinking:
``What are the primitives, what are their means of combination,
and what are their means of abstraction?''

To follow these words we would first have to look at
the \glspl{construct} of \gls{js}.
Then we can determine in what ways these \glspl{construct} can be combined.
Finally we could look at the means of abstraction \gls{js} offers us,
but unfortunately there is not enough time
to handle that aspect in this research.

\subsection{Relative execution time}

The \gls{rel-exe-time} of a given \gls{code} can be measured mathematically.
As explained by \citet[Chapter 4]{goodrich2008data}
the \gls{rel-exe-time} can be expressed using seven functions
(constant, logarithm, linear, n-log~n, etc.).
To determine with which function
the gls{rel-exe-time} of a given \gls{code} can be expressed
one has to look at the \glspl{construct} used in that \gls{code}.
For each \gls{construct} the \gls{rel-exe-time} has to be determined.
Then you can calculate the \gls{rel-exe-time}
of the combination of those functions
(i.e.\ the combination of the \glspl{construct} that make op the \gls{code}).

This method would be a good way to see
if the \gls{student} has used an algorithm
that is too complex for the given \gls{exercise}.
If, for example, the \gls{solution} of the \gls{tutor} executes in n-log~n time
while the \gls{code} of the \gls{student} executes in quadratic time,
\gls{feedback} could be given to the \gls{student}
that he should use a more efficient algorithm for that \gls{exercise}.
The question would be:
how to give the \gls{student} more concrete \gls{feedback}?
Instead of just telling him he can do better
it would be constructive to point him in the right direction.
However, this would require more knowledge of the particular \gls{exercise}
and the difference between the \gls{solution} of the \gls{tutor}
and the \gls{solution} of the \gls{student}.

An implementation of this method would require more than
determining the \glspl{construct} used in a given \gls{code}
and how they are combined.
When a loop is used, for instance,
you would need information about run time variables
to determine how many times the loop would be iterated over.
In other words: looking at the \glspl{construct} is not enough,
you would need a good understanding of the algorithm used
and the problem that is being solved.
This understanding might be provided
by letting the \gls{tutor} specify information about the algorithm used
when creating an \gls{exercise} in the \gls{examiner}.
However further study will be required to figure out
how this would best be implemented,
and what information would be required exactly from the \gls{tutor}.

\subsection{Language constructs}

When looking at the meaning of the \gls{code}
you can distinguish different \glspl{construct}.
When you look at an \gls{exercise},
it might be the case that a good \gls{solution}
should contain one or two specific \glspl{construct}.
That means that we can look for those structures
in a \gls{student}'s \gls{code}.
If they are not there, or if there are too much of them,
the \gls{student} can be given \gls{feedback} mentioning this fact.
In order to be able to give the \gls{student} a more profound explanation
of why his \gls{code} is not optimal we would need input from the \gls{tutor}.
Because our goal is to relieve the \gls{tutor} from too much work
we could try and implement this feature in a similar way
\citet[Section 3.2]{watson2011learning} proposed in their article.
They describe a way to provide \gls{feedback}
on how to correct a given \gls{code} that generates an error
(halting on execution).
The first time the system encounters a specific error
it has to request the \gls{tutor} for \gls{feedback}.
The system saves that \gls{feedback} into a database along with the error.
On any subsequent occurrence of that same error
in any of the \glspl{student}' \gls{code}
the system retrieves the \gls{tutor}'s \gls{feedback} from the database.
It does not need to bother the \gls{tutor} any more,
while the \gls{student} receives valuable \gls{feedback}.

We would need to determine first what \glspl{construct} are important
for a specific \gls{exercise}.
When a \gls{student} uses a for loop where the \gls{tutor} does not,
that would be an interesting case.
But when a \gls{student} uses one more simple assignment statement
than the \gls{tutor} does,
that will probably not be very important for instance.
We can determine the set of \glspl{construct} used in the \gls{code},
by type and by quantity.
And compare that with the \gls{solution} of the \gls{tutor}.
This gives us a set of \glspl{construct} the \gls{student} did not use
while the \gls{tutor} did,
and a set of \glspl{construct} of the other way around.
After filtering these two sets of \glspl{construct} for important ones
we have something to give \gls{feedback} on.
Now we can use the same mechanics as \citet{watson2011learning} described.
We look in the database for the \gls{feedback}
on a particular \gls{construct} that was used --- or not used ---
and show that to the \gls{student}.
In case no \gls{feedback} was found in the database,
the \gls{tutor} is requested for input.
In that case the \gls{student} will not receive immediate \gls{feedback}.
But after the \gls{tutor} has provided \gls{feedback} to the system
all other \glspl{student} who run into the same problem
will receive the \gls{feedback} immediately.

Interesting use of this method for later implementation can be the comparison
--- not only of the a \gls{student}'s \gls{solution}
with the \gls{tutor}'s \gls{solution} ---
of a \gls{student}'s \gls{code} with the \gls{code}
of previously submitted \gls{code} of other \glspl{student}.
This might give the \gls{student} \gls{feedback} on how he is doing
relative to his fellow \glspl{student}.

Research would be needed to see if this would work
the way I describe it here.
Would it be feasible to determine how to rate the \glspl{construct}
of the \gls{js} language?
And would the system find all the problem cases or will it miss some,
or find cases that are of no importance?
Besides that, this functionality would have to be created from the ground up
because it is quite specific and I have not found something like it
for \gls{js}.

\section{Comparing code}

The \gls{code-layout} is an absolute check to see
if the \gls{student}'s \gls{code} adheres to a well defined set
of rules of best practices in coding.
The \gls{code} either fails on some points
or it is completely laid out according to the layout definitions used.
How you go about giving \gls{feedback} when some rules are broken
is another question.
The \gls{examiner} could say the \gls{code} is incorrect
or more gently tell the \gls{student} that it could be better.
It could clearly point out where the shortcomings are
or simply tell the \gls{student} to do better.
The point is that the rules of \gls{code-layout} can be well defined,
how strict you hold the \gls{student} to it is a matter of \gls{feedback}.

A different matter are the other metrics
for \glslink{code-quality}{code} and \gls{exe-quality}.
They do not simply give you an answer whether the \gls{code} is good or not.
It is more of a gray area and you would need to determine
when \gls{code} is good, when it is acceptable
and when it would need improvement.
These acceptability margins will be different for each \gls{exercise}
(e.g.\ a more complex assignment should allow a higher cyclomatic complexity).
The \gls{tutor} could be asked to specify these margins
when creating an \gls{exercise},
but the \gls{tutor} might struggle to determine good margins
for the \gls{exercise} at hand.

The great thing is:
we do not need to determine the quality of a given \gls{code} on its own.
We can compare it to a well defined base \gls{code}
that is assumed to be correct and efficient (i.e.\ an acceptable solution),
by asking the \gls{tutor} to write a \gls{solution}
to the \gls{exercise} himself.
The \gls{examiner} could determine a baseline for the different metrics
by calculating the metrics for the \gls{tutor}'s \gls{solution}.
That means that we do not need to decide
how many lines of code would be too much,
or what cyclomatic complexity would make the \gls{code} too complex.
We can just compare the lines of code,
and the Halstead complexity of a \gls{student}'s \gls{code}
with the \gls{solution} of the \gls{tutor}.
That way the system would have a good base for which it can say:
when I get these results (or better results)
for the metrics on a given \gls{code}, the \gls{code} is good,
otherwise the \gls{code} could be better.
Preferably you would still want some form of margin,
where the results of the metrics only slightly below those of the \gls{tutor}'s
are as acceptable as the \gls{tutor}'s.
Determining how those margins can best be calculated
is subject to further research.
We do have to determine however: how bad is the \gls{code}
if it is not as good as the solution of the \gls{tutor}?
There should probably be some margin of acceptance.


\section{Conclusion}

While no one method is perfect it would be best to combine several methods
to determine the quality of the code and of its execution.
Looking at \gls{code-layout} is a good starting point
and will be easy to implement because many tools exist
for performing these checks for \gls{js-code}.
The other methods will require a baseline
which can be obtained by analyzing the \gls{tutor}'s \gls{solution}.
This won't be a problem to implement as well
but would require further research and development
for determining the right margins of acceptance of these metrics.

The maintainability metrics are well defined
and have been implemented in different tools.
Therefore these are also a good candidate to implement in the \gls{examiner}.
Some extra research might be necessary though
to find the right implementation, use and combination of these metrics
for best representing the quality of the \glspl{student}' \gls{code}.

Examining \gls{exe-quality} of a given \gls{code}
would be an interesting area to look into,
but will take more time and effort to implement.
To calculate the \gls{rel-exe-time},
good understanding of the \gls{code} and the algorithm are needed.
Also --- because of its complexity --- no ready to be used tools exist
for executing this specific method.
Serious research and development will be needed to be able to implement it.

Building up a database with specific usages of \glspl{construct}
for an \gls{exercise} along with helpful \gls{feedback} for a \gls{student}
seems an interesting solution.
The \gls{examiner} will be able to generate valuable \gls{feedback},
simply and effectively by asking the \gls{tutor}.
At the same time the \gls{tutor} will be relieved from too much work
because the system would save that \gls{feedback} into a database.
Further research is needed to determine the best way of
extracting the important \glspl{construct} from \gls{code}.
Development is needed to implement this functionality
and make sure the \gls{feedback} from the database
will be supplied at the right times.

