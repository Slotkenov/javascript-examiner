% verslag analyse onderzoekcontext, plusminus 5 pagina's

\section{Introduction}
% Research group: IDEAS
% Tool: Ask- Elle
% Focus: research question 2 and 3: easily adding and finetuning exercises by a 
% tutor 
% Corresponding requirements:
%  - The tutor should be able to create an exercise (...)
%  - Code Layout definition
%  - Programming guidelines definition

Often, science itself is characterized/designated as being the root of 
scientific breakthroughs/evolution. However, the starting point of the 
\gls{examiner} project has been pragmatic rather than scientific. 
The urge to reduce labor intensity while
maintaining or even improving the extent of providing \gls{feedback} has little 
scientific value by itself. From this perspective, it may be argued there is 
no scientific context, or at least no computer science related context. 
This does by no means degrade the scientific value of this project, it just 
denotes the scientific research context is to be found elsewhere. 

The domain research has a practical focus for the main part. The importance of
this approach is clear, when keeping the goal (a functioning prototype
of the Javascript-Examiner) in mind. Where the <Ronalds Artikel> and Feedback
papers are focused on some aspects of the domain, the Semantics article revealed
a more fundamental question. In order to get beyond the checking of rudimentary
characteristics (like functionality, format, syntax) it is necessary to 
reason about \gls{js}, or even programming languages in general, itself. Of
course, it is way beyond our limits to pursue this reasoning to it is 
foundations. Luckily, this is no unexplored domain.


The main stakeholder of this project directed us towards the IDEAS framework,
which is the foundation for a set of domain reasoners. 
After broadly studying \citep{gerdes2012ask} and \citep{heeren2010specifying}
their approach (with regards to semantic analysis) 
revealed some possible relevant pointers. Relevant because of their goal (alike)
and method (different) as well on their approach of systematic/semantic analysis.
% Two way: Scope, Extent (GOAL)
% With regards to Semantic Analysis

%Some info about IDEAS from the perspective of the Examiner-Project
-Same high level goals
-Practical approach (prototypes)
-Differences 
%End Some info about IDEAS>>

We found Heeren, B. willing to discuss some aspects of the IDEAS project. As 
this project -probably deliberately- has been presented to us as being the 
start of a new, autonomous from any existing project, the scope and extent for
the consult has been fairly general. Our interest has been merely about the 
conceptual considerations and scientific foundations related to IDEAS, 
rather then the pure technical specifications. This focus led to an, in our
experience, interesting and fruitful consult. Within this demarcation we picked
two distinct topics: 
	* Project Goals and Approach (Similarity --> Distinct)
	* Systematic / Semantic Analysis 

\section{Consult}
%Verslag consult.
%Vragen + Antwoorden uitwerken:
%Questions
%  - Is it possible to get insight in the actual process of exercise creation?
%  - How many exercises have been created by tutors? 
%  - How is their reception of the exercise creation process? (research already 
% in progress?)
%     * What are the difficulties they experience?
%     * Are their didactic aspects they can't, but would like to introduce? 
%     * What features are used most? (i.e. Global vs Location specific)
%  - What were the arguments to work with model solutions? Have other 
% possibilities
%    been considered? (i.e. test suites)
%  - Is there a limit for the magnitude of an exericse? What are implications 
% (towards possibilities, scope etc.)?
%  - Would it be possible to create a set of exercises in a way there is no need
 % for additional material?
%  ? Are there indicators for a tutor on which constructs feedback annotations 
% are needed? How does
%    the retrieve these indicators?
%  - Is there a size limit or minimum for a  provided set of model solutions?
%  - The correctness is based on equality. Is the current measure method to 
% determine (in)equality rightly balanced?
%  - Is it possible to get a demo on actually adding a simple exercise?
%  - Is Ask-Elle flexible in usage? (mainly in the way exercises are added)
%  - How to proceed if there are 3 model solutions,
%    and two solutions need feedback on the same particular construct,
%    should this feedack annotation be global?
%  - How does Ask-Elle make a distinction between the different solutions (from 
% optimal to suboptimal)?
%    @TODO: meer specifieke vragen (met name hoofdstuk 7 + hoofstuk 5 (feedback 
% scripts))
%  - How are the feedback texts generated?
%    And how do you choose which feedback to show
%    according to the step a student took.
%  - Could the normalization rules be easily ported to JavaScript?
%  - Could the Haskell service be modified to be used for JavaScript code?
%  - Or could the functionality of the Haskell service be ported to JavaScript?
%  - How dificult is it for a tutor
%    to provide the right annotations to a model solution?
%  - Could the annotations used by Ask-Elle be used in the same way
%    for a non-functional language (e.g. JavaScript)?


% Topics of discussion:
% Formative / Summative distinction
% Ask-elle (and alike IDEAS implementations) has a more summative approach, 
% whereas the focus for the Javascript-Examiner for a great part has a formative n
% Nature.
% Normal form
% Deceitful problem (equality)
% Semantic preserving variations
% http://www.cs.uu.nl/research/techreps/UU-CS-2014-026.html
 % (article Hieke Keuning)
% http://dspace.ou.nl/bitstream/1820/5388/1/INF_20140617_Keuning.pdf 
% (thesis Hieke Keuning)
% Compared to dynamic analysis an inverted approach
% Model solutions
% In the IDEAS approach, more model solutions are required as the exercise expands
% due to the search space (tending to exponential)
% Agorithmic variations (i.e. Quicksort vs Bubblesort)
% Static comparison
% Inverted use of Plagiarism detectors (Juriaan Haring in Java (UU)
% Div tools
% Edit distance

% Implementation 
% Services based
% Modular
% Easily expandable / adaptable
% Solutions
% Collect solutions written by Students for testing the Javascript-Examiner

% Abstract Syntax Tree
% Possibilities to clean the Abstract Syntax Tree at some stage, in order to be 
% able to achieve more accuracy when it’s used for i.e. Semantic Analysis

\section{Semantic Preserving Variations}

% ????? In hoeverre dit uitwerken ????? Vraag voor Annemiek. ????? Met name 
% gezien het praktische uitgangspunt van Harry en Sylvia.


% From Xu and Chee, explained with the application of IDEAS based reasoner 
% Hieke Keuning
% 1 	Different algorithms
% 2 	Different source code formats
% 3 	Different syntax forms
% 4 	Different variable declarations
% 5 	Different algebraic expression forms 6 Different control structures
% 7 	Different Boolean expression forms 8 Different temporary variables
% 9 	Different redundant statements
% 10 	Different statement orders
% 11 	Different variable names
% 12 	Different program logical structures 
% 13 Different statements

% Examples:
% SPV1 - Different algorithms (i.e. sorting)
% 	SPV2 - Code format (spaces, indentation, comments)
% 	SPV3 - Syntax forms (if(i) {x = y} else{x = z} or x = (i) ? y : z)
% 	SPV4 - Variable declaration (method / block)
% 	SPV5 - Algebraic expression forms (+ vs --)
% 	SPV6 - Control structures (branching, iterating, traversing)
% 	SPV7 - Boolean expression forms (x == false vs !x (SPV3))
% 	SPV8 - Temporary variables
% 	SPV9 - Redundancy / Code for debugging
% 	SPV10 - Different (statement) order
% 	SPV11 - Diferenent variable names (parameters, functions)
% 	SPV12 - Different logical structures
% 	SPV13 - Different statements

% three forms of representation:

% 	1 - Source Code
% 	2 - AST
% 	3 - Augmented Object-oriented program dependence graph (AOPDG)

% Source-to-specification approach:
% 	-Consistent with Research on program understanding
% 	-Match student code with templates with corresponding specifications OR
% 	-Match student code with specifications like:
% 		*structures
% 		*schemata
% 		*plans
% 		*assertions
% 		*processes
% 	-Moeilijk generaliseren voor alle typen programmeertalen

% Specification-to-specification approach:
% 	-Derive formal specifications from student and reference code
% 	-Not possible for many languages

% Source-to-Source approach:
% 	-performs automatic reasoning at the level of program dependence graphs
% 		(FDG: semantic representation for programs, but no formal specification)

% The Xu and Chee approach:
% 	-Model programs are used as input to the diagnosis of student programs. 
% 	-Comparing student program with model program after standardization 
% (by program transformations)
% 	-transformation-based diagnosis
% 	-Simpler to make use of in practice

% 	-Program standardization and semantic-level program matching
% 	-No intentional checks
	% -Intraprocedural diagnosis method (class hierarchies and inter-procedural 
	% variations are not included)
% 	-Different model programs used to diagnose students program
% 	-correct = being equivalent after standardization
% 	-semantic differences indicate errors

% 	-Computational Semantics vs Operational Semantics


% 	-Standardize programs (representation)
% 	-Match them



 %  Program transformations:

 %  	Basic: (SPV3,4,5,6,7)

 %  	-Statement separation	(remove cascaded messages i.e. var a, b, c = 5);
 %  	-Temporary declaration standardization (declaratie variabelen op bepaalde plaats);
 %  	-Algebraic expression standardization (rules voor notatie berekeningen (normaalvorm));
 %  	-Controle structure standardization (??);
 %  	-Boolean expression standardization (rules voor notatie boolean expressies !x --> x == false);


 %  	Adv:

 %  	-Forward substitution
 %  	-Dead code removal

 %  	3 criteria:

 %  	-Applicable to programs at source code level
 %  	-Covergant (avoid cyclic triggering)
 %  	-Intra-procedural (as limited to intra-procedural)



 %  AOFFG (AUGMENTED Object-Oriented Flow Graph) Program Respresentation

	% -PRG: Program Representation Graphs (Static compile-time semantics)
	% 	*SSA: Static-Single-Assignment
	% 	*PDG: Program Dependence Graphs
	% 		*CDS: Control Dependence Subgraph
	% 		*DDS: Data Dependence Subgraph
	% 	*OPDG: Object-oriented PDG
	% 		*ODS: Object Dependence Subgraph
	% 







	
\section{Conclusion}