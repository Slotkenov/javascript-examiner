% report research context

\section{Introduction}
% Research group: IDEAS
% Tool: Ask- Elle
% Focus: research question 2 and 3: easily adding and fine tuning exercises by a 
% tutor 
% Corresponding requirements:
%  - The tutor should be able to create an exercise (...)
%  - Code Layout definition
%  - Programming guidelines definition

The starting point of the 
\gls{examiner} project is pragmatic rather then scientific. 
The urge to reduce labor intensity while maintaining or even improving the 
extent of providing \gls{feedback} has little scientific value by itself. From 
this perspective, it may be argued there is no scientific context, or at least 
no computer science related scientific context. This does by no means degrade 
the scientific value of this project, it just denotes the research context is 
to be found elsewhere. 


The domain research has a practical focus for the greatest part. The 
importance of this approach is clear, when keeping the goal (a functioning 
prototype of the \gls{examiner}) in mind. Whereas the Design Recipe and 
Feedback papers are merely focused on some aspects of the domain, the 
Semantics paper reveals a more fundamental question. In order to get beyond 
the checking of rudimentary characteristics (like functionality, format and 
syntax) it is necessary to reason on \gls{js}, or even on programming 
languages in general, itself. Of course, it is way beyond our limits to pursue 
this reasoning to its foundations. Luckily, this is no unexplored domain.


The main stakeholder of this project directed us towards the IDEAS framework, 
% reference
which is the core for a set of \glspl{domainreasoner}. After broadly studying 
\citep{gerdes2012ask} and \citep{heeren2010specifying}, their approaches (with 
regards to semantic analysis) revealed some possible relevant pointers. The 
IDEAS framework is used to generate feedback on input from a specified domain. 
The domain may vary, as the first article is about \gls{haskell}, whereas the 
second is about rewriting strategies for logic expressions.


Both articles describe research projects that focus, like our project, on
providing feedback for (academic) students. Hence, the former article
strengthens the relevance, in putting the usability and maintainability for 
tutors as its central research question. Thus, the goals to our project are 
very much alike. 


When considering their approach, there is one fundamental difference. The IDEAS
framework is designed to provide stepwise feedback. Instead of reviewing a
submitted solution as being complete, it is possible to review a partial 
solution. This feature is deliberately not part of our project. Still, it is 
not unthinkable that there might be successive projects to extend the 
\gls{examiner} with this feature. Like our approach, the developed domain 
reasoners are built as prototypes at first. In order to be able to provide 
feedback on partial solutions, a thoughtful and sophisticated reasoning on the 
domain specific language semantics is required.


We found dr. B.J. Heeren willing to discuss with us some aspects of the IDEAS 
project. Since our project has been presented as being the start of a new and 
autonomous project, the scope and aim for the consult is fairly general. 
We are primarily interested in the conceptual considerations and scientific 
foundations (in particular on semantic evaluation) rather than focusing on the 
pure technical specifications. Eventually, this interest led to an, from our 
experience, interesting and fruitful consult. 


In what follows, the consult will be described. As will become apparent, some 
interesting topics will pass. As a result, a brief examination of one of these 
topics will follow. In conclusion, the consult and examination will be linked 
to our project. Furthermore, some interesting topics will be hinted for 
possible succeeding projects.


\section{Consult}
Due to the nature of our project, the outline and boundaries of this consult 
were not clear beforehand. To prevent a lack of scope, we specified two 
central topics, namely semantic evaluation and the usability of the reasoners 
from a tutor perspective. The meeting with Heeren was very open and fluent. We 
did not strictly follow the order of our prepared questions during the 
conversation. Still, at times the 
conversation spirited away, we used the questions to get back to the main 
topic. Since we did not literally recorded the consult, the resume below is an 
interpretation based on the prepared questions, the consult itself and the 
keynotes taken.

%Formative - Summative
After a short introduction on the IDEAS framework and our project, the 
discussion started with a demarcation posted by Heeren, in putting the 
formative approach of the IDEAS related projects against the summative 
approach of our project. This distinction is the result of the difference in 
approach. The extent to which our project really is summative, is withheld by 
the way it will be used eventually. %Work out argument 
Still, it seems to be fruitful to keep this distinction alive for now, as the 
central point (no partial solution feedback functionality in our project) is 
striking. 


\subsection{Semantic Evaluation} %Please review this subtitle 
At this point, the conversation shifted to the topic of semantic evaluation.
Heeren introduced an overview of the technique to bring code back to a
normal form. This technique abstracts from the particular code, by replacing 
constructs. For example, a \emph{for loop} can be replaced by a 
\emph{while construct}, without 
changing the execution result of the code. In this way, two solutions can be
compared, even if the first uses the \emph{while} construct, and the second a 
\emph{for loop}. At first, this seems very fruitful to use, but Heeren made clear
that the search for a normal form is quite hard. Reasoning toward a normal form
is reasoning about equality. Under what conditions can be stated that two or 
more statements are equal. ``\,A deceitful problem''\,, according to Heeren. To 
get more insight in the complexity of this topic, an article about one of the
IDEAS based reasoners was recommended to us. \citep{keuning2014strategy} 
stresses the equality problem by reviewing her implementation (and in
doing this the framework as a whole) to the exhaustive set of normalization 
techniques under the name of \gls{spv}. This set, as argued in 
\citep{xu2003transformation}, covers a complete spectrum of normalizing 
techniques on a semantic level. These articles could be the appropriate 
starting point if one wants to add semantic normalization functionality 
to the \gls{examiner}.


After this theoretical introduction, some practical pointers were given, that
could be helpful as well, when starting to experiment within this area. 
Thorough analysis of (generation of) \glspl{ast} might discern opportunities 
to get rid of some variations, and as such provide a first implementation of
normalization. %Maybe refer to next chapter 
Furthermore, diving into techniques that deal with static comparison or static
testing might be fruitful, as there is (semantic) reasoning required in that
area as well. Finally, the inverted use of plagiarism detectors could be 
interesting. Inverted use, because the detectors are designed to deal with input 
that is ought to be different from each other (the input to be tested is 
reviewed against all existing documents), whereas we expect and search for 
similarities to, for example, match a solution with (one of the) example 
solutions. 


Packed with both a theoretical introduction and practical handlebars on
Semantic Evaluation, the topic changed to our second general question.

\subsection{Create an exercise}
The answer to our main question concerning this topic was clear. At this moment
there are no results from usage of the examiners by tutors that did not
actively participate in the development of the reasoners. Looking at the 
complexity and theoretical approach, this seems reasonable. Still, it stresses
the ability to determine the value of the approach, when comparing it to our 
own project, as our success heavily depends on the usability by the tutors. 
Apart from the lack of test results, Heeren did outline a possible time
consuming factor for tutors, when using the reasoners. As it is hard or even 
impossible to normalize algorithmic variation, and the diversity in usage of 
algorithms tends to grow when the size of an exercise and solutions extends, 
the only way of dealing with this diversity is adding more model solutions 
(which implements different algorithms). 
%May be add example (Quicksort vs Bubblesort oid)
As the space of possible variation grows, the need
for model solutions grows, with a factor tending to exponential. This can 
quite quickly stress the reduction of the tutors labour intensity, which is one 
of our main goals. Therefore, it's quite important to keep this goal in mind
while designing and developing the examiner, as new features might have 
unwanted implications to the time and efforts required to successfully deploy 
the \gls{examiner} or new \glspl{exercise}.

\subsection{Implementation Pointers}
During the consult, we recognized Heeren had gained a lot of research and 
development experience. To learn from him, we concluded the consult by 
asking him what pointers we should take in mind while developing the 
\gls{examiner}. According to him, we should adapt an approach which allows for 
easy expanding and adapting. As our project might be only a first step in a
series of \gls{examiner} related projects, this characteristic is of great 
importance. He advised us to develop service based or at least in a modular 
way. On top of this, he stressed the importance of relevant test data. It
could be a pitfall to only use our own written exercises and solutions, because
it's not unlikely that solutions from other students are fundamentally 
different when it comes to the means of properly analysis and review.

\subsection{Resume}
Enthusiastic and enlightened, we thank Heeren for his help and contribution.
While reviewing the consult, we recognized our shortcomings in the understanding
of problems related to semantic equality and \gls{spv}. Even if we 
don't implement related functionality during this project, it's still required
to grasp the fundamentals in order to be able to design and develop an 
appropriate foundation that is enabled to actually implement functionality 
related to semantic evaluation. As a result, the \gls{spv} has been briefly
analyzed and the results are included in the next section.

\section{Semantic Preserving Variations}
\gls{spv}, as outlined in \citep{xu2003transformation}, are the result of code
analysis from a conceptual point of view. As the researchers themselves were
actively participating in the discourse of programming education, their goal
was related to ours. Their approach was quite different. This may have to do
with the more scientifically centered base, but its likely this also has to do
with the fact that there were not yet as much relevant research results 
available in the late nineties. 

The article begins with the positioning of their approach, in 
demarcating the used technique from other code analysis techniques. These other
approaches are:
\begin{description}
	\item[Source to Specification] This approach is consistent with the (at that
	time) research on program understanding. The technique aims at matching a
	\gls{solution} with corresponding (specific to the related exercise) 
	specifications or with general specifications like:
	\begin{itemize}
		\item Structures
		\item Schemas
		\item Plans
		\item Assertions
		\item Processes
	\end{itemize} 
	\item[Specification to Specification] In these approach, the \gls{solution} 
	itself is first converted to a specification, and this specification is 
	compared with an model specification. According to the article, the usage
	of this approach is limited to only a few programming languages.
	\item[Source to Source] This approach is the base for the \gls{spv}. It aims
	at comparing the student \gls{solution} with one or more model \gls{solution}(s).
	To enable comparison, the \gls{code} is transformed from the actual code to
	\glspl{ast} and program dependence graphs. These graphs are like formal 
	dependence graphs, without being a formal specification.
\end{description}
According to the Source to Source approach, model solutions are used as 
instrument to examine student solutions. In order to be able to do so, both 
solutions are transformed. In general, these transformations are the key to
the success of this approach, so the article describes the distinct 
transformations techniques in great detail. As we have no intention to implement
these techniques in the current project, we will only focus here on the 
conceptual results of these transformation. The actual results are the 
elimination of distinctions in the solutions, while maintaining the semantic
integrity of the code. The conceptual result are solutions that are adjusted in 
a manner they can be compared. In what follows, an overview of the
transformations is given, as much as possible accompanied with examples that 
correspond with \gls{js-code}. 

%Eventueel hier wel een paragraaf over de technieken achter de transformaties:

%  Program transformations:

 %    Basic: (SPV3,4,5,6,7)

 %    -Statement separation (remove cascaded messages i.e. var a, b, c = 5);
 %    -Temporary declaration standardization (declaratie variabelen op bepaalde plaats);
 %    -Algebraic expression standardization (rules voor notatie berekeningen (normaalvorm));
 %    -Controle structure standardization (??);
 %    -Boolean expression standardization (rules voor notatie boolean expressies !x --> x == false);


 %    Adv:

 %    -Forward substitution
 %    -Dead code removal

 %    3 criteria:

 %    -Applicable to programs at source code level
 %    -Covergant (avoid cyclic triggering)
 %    -Intra-procedural (as limited to intra-procedural)


%  Program transformations:

 %  AOFFG (AUGMENTED Object-Oriented Flow Graph) Program Respresentation

  % -PRG: Program Representation Graphs (Static compile-time semantics) (OFG: Object-oriented Flow Graph for OO)
  %   *SSA: Static-Single-Assignment
  %   *PDG: Program Dependence Graphs
  %     *CDS: Control Dependence Subgraph
  %     *DDS: Data Dependence Subgraph
  %   *OPDG: Object-oriented PDG
  %   *PDG
  %     *ODS: Object Dependence Subgraph
  %     *ODG: Object based dependence graph   
  %     *
  % 
  %   *AOPDG (Augmented Object-oriented Program Dependence Graph) = AOCDS (Augmented Object-oriented Control Dependence Subgraphs) + AODDS (Augmented Object-oriented Data Dependence Subgraphs)(contstructed from AOCDS)
  %   

% Program comparison

  % Partitioned graphs: if vertexSP and vertextMP in same set: statement in vertices semantically equivalent
  % 3 result maps: equivalent map, textual difference map, unmatched map
  % 

% Error detection

  % Problem: Operational semantic differences rather than computational semantic differences
    % --> solution: variation learning process to detect unsystematic semantics-preserving variations

\subsection{Variations} %TODO: The code examples within this chapter should be
% examined and put in 'blocks'.
There are thirteen distinguished \gls{spv}. All variations will be mentioned and
briefly described, with reference to possible usage in \gls{js-code}. 
The extent of a variation is not always clear. 
\begin{description}
	\item[Algorithms] This is alike within each programming language. For example,
	when there is sorting required, there are several algorithms that can be 
	applied.
	\item[Source code formats] As \gls{js} is a very tolerant language, the
	code formats can differ a many levels. For example, the usage of space can be
	omitted at many places: if( or if ( does compute the same thing, another
	striking example is the non-committal usage of ``\,;''\, to mark the end
	of a statement. In our project we already implemented a code style checker, 
	which enforces a particular code format. Another approach to bypass this 
	variation is to parse an \gls{ast}.
	\item[Syntax forms] The tolerance of \gls{js} is relevant for this variation
	type as well. Writing a simple conditional assignment can be done in many 
	ways:
	\begin{itemize}
		\item if(i) x = i; else x = 0;
		\item if(i) { x = i;} else { x = 0;}
		\item x = (i) ? x = i : x = 0;
		\item x = i \textbar \textbar 0;
	\end{itemize}
	\item[Variable declarations] For the declaration of variables there are 
	several options. Usage of the keyword ``\,var''\, or not (which declares a global 
	var, declaring variables in a sequence (i.e var a, b, c), using the keyword 
	``\,const''\, for constants.
	\item[Temporary variables] Temporary variables can be declared when used, but
	also in a \"parent\" block, and then reassigned when needed. As mentioned in
	the former item, variables declared without the ``\,var''\, keyword, can be used
	anywhere in the program, even if declared in a block.
	\item[Algebraic expression forms] This is much alike for all programming
	languages, in any case they share the same operators. A very simple example is
	writing -- against +.
	\item[Redundant statements] Statements that are identical have the same result
	when executing the code.  
	\item[Order statements] The order of statements is in many case arbitrary, for
	as long they are not dependent on each other in between.
	\item[Naming variables] As names of variables can be seen as labels, they do
	not determine the execution. In case of \gls{js}, it's possible to use an
	existing module called Uglify, which replaces variable names with externally
	specified ones.
	\item[Logical structure] %TODO: Add description + example
	\item[Statements]	%TODO: Add description + example
\end{description}


As became clear, some variations can already be tackled with the techniques
currently used in the \gls{examiner}. This might appear to be an indication of 
the applicability of techniques to overcome the other variations. We think this 
is misleading. The variations are exhaustive, so it is by no means exceptional 
that some variations can be dealt with very easily. Besides, as long as there are
variations that can't be recognized, it's quite hard or even impossible to make
claims about the quality of a \gls{solution}. 
	
\section{Conclusion}
When determining the research context, the primary stakeholder directed us 
towards related projects that for great part deal with the semantics related
evaluation of \gls{code}. This became the main topic during the literature 
study, both before and after the consult, as the central topic for the consult
itself. After studying \citep{gerdes2012ask} and \citep{heeren2010specifying},
it became clear the IDEAS framework could provide us with handlebars for our
research, and a consult with one of the stakeholders could be very fruitful. 
After the (reflection on) the consult, we can acknowledge this supposition.


Heeren gave us insight in the complexity of semantic analysis when it comes to
programming languages. He directed us via \citep{keuning2014strategy} to 
\citep{xu2003transformation}. Based on these articles, we were able to get a
more structured insight on this complex field. This insight can help us to 
design and develop the \gls{examiner} in a way that facilitates implementation 
of semantic examination.


With regards to this design and development, Heeren shared his experience and
gave us advice. Developing in a service based and modular way enables expansion
and adjustment in succeeding projects. As we discovered in the examination of 
\gls{spv}, it could be beneficial to make artefacts generated with one 
service accessible for a successive service. For example, when performing a
syntax check, an \gls{ast} is generated. In a later project, a service
might be added that needs an \gls{ast} in order to work. Now, we take this into
account while developing the \gls{examiner}. Furthermore, the all ready 
implemented code style check functionality might be used as starting point to
overcome some of the \gls{spv}. In a more general way, these insights can be
used in successive projects, to determine which direction to take, and in case
of semantic analysis, offer some guidance.


Besides the semantics related enlightenment, the practical advices and insights
are taken into account as well. Now, we consider the usage of the 
\gls{examiner} by actual \glspl{tutor} even more important, as the 
implementations of the IDEAS framework based domain reasoners have shown the
difficulty to achieve this. Related to this ``\,production phase mindfulness''\,, 
the usage of actual student solutions to test the \gls{examiner} has gained 
priority. 


Before developing functionality that is able to overcome the mentioned 
variations, thorough research on the practical demand should be performed. 
As we move towards a production ready state for the \gls{examiner} during
this project, we hope to pave the way (in exposing the \gls{examiner} to 
students) to get insight in the degree to which this demand exists, and to what 
extent this kind of examination actually benefits to the reduction of labour 
insensitivity as to support of programming language education in general, and 
learning \gls{js} in particular.