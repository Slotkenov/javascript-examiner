% This chapter should discuss the global architecture of the whole app.
% It should not go into detail or implementation specifics.
% The chapters 'frontend', 'backend' and 'maintainability'
% should go into more detail and the implementation specifics.
%
% - Separation between domain, checks, RESTful server and frontend.
% - Checks as separated modules, but with the option to contact the domain.
% - Frontend with isolated elements and option to contact RESTful server and
%   checks.

The \gls{examiner} consists of four parts.
First of we have the domain we are working on, consisting of, among others,
% Hier verwijzen naar het domain model
tutors and students, the exercises, solutions
and the data that comprises these entities.
Secondly we have a control layer,
which acts as a separation between the domain layer and the application.
Then there are the various \glspl{check} for investigating \gls{js-code}.
And finally, since we are dealing with a web application,
we have a distinct client side.
We will go into the details of each part in the sections below,
where we will discuss their part in the global architecture of the system,
and how the communication flows between each part. For a graphical 
representation of the architecture, see appendix \ref{app:architecture}. A 
domain model can be found in \ref{app:domain-model} whereas an activity diagram 
that visualizes the code submission is presented in \ref{app:activity-diagram}

\section{Control layer}
The control layer is the server side,
in the client-server architecture common to web applications.
It handles clientside requests
for retrieving or modifying a part of the domain.
Besides facilitating communication between the client side and the domain
it also does other things, like sending an email.

\section{Checks}
The \glspl{check} are the modules that perform the examination of \gls{js-code}.
In order to make it easy to extend the system with more \glspl{check} in the future
we placed the \glspl{check} in relative isolation to the other parts of the system.
In the same way that the client communicates with the control layer,
the client can also communicate directly with one of the \glspl{check}.
Each \gls{check} in its turn communicates back to the client
and can reach out to the domain if necessary.

As for the flow of the communication
the \glspl{check} have the same place as the control layer.
In that manner they each are a specialized mini control layer on their own.
We deliberately pulled them apart from the control layer though
in order to keep them as much plug-and-play as possible.
This way the \glspl{check} are not limited
to the implementation we choose for the control layer.
While it is possible to place a \glspl{check} on the same server,
and write it in the same language as we choose for the control layer.
With this architecture it is certainly possible as well
to write the \glspl{check} in an other language
and even run it on a different server.

The only coupling the \glspl{check} have is with the client and the domain.
The client controls the logic for contacting the various \glspl{check}.
That logic would need to be updated when a \gls{check} is added.
The \glspl{check} also need access to the domain
and need to be able to access
the specific data storage implementation we choose for the domain.
The domain only needs to be altered if a \gls{check} expands on the domain model.
This way the \glspl{check} are decoupled from the control layer
and its implementation which improves the ability to write, and change 
the \glspl{check} for the \gls{examiner}. 

\section{Client side}
Because we are building a web application
we have to deal with the browser side of things.
Thus we have a client side in our application.
To provide a rich user experience we created a single page application.
No page reloads will occur
and communication with the server will be done through
XHR requests\footnote{\url{https://developer.mozilla.org/en-US/docs/Web/API/XMLHttpRequest?redirectlocale=en-US&redirectslug=DOM\%2FXMLHttpRequest}}.
The client will be responsible for the user interface
and the interaction with the user.
It will also carry out the logic directly coupled to the user interface.
For more domain specific logic
the client side will contact the control layer on the server side.
As well as for performing the various \gls{js-code} \glspl{check}
the client side will contact the \glspl{check}.

\section{Discussion}
We will be laying the groundwork for this project
with just a few of the functionalities it requires.
Other teams will pick up where we left and complement this software project.
Because of this we decided to use a service oriented approach.
Each functional requirement for examining \gls{js-code}
will be encapsulated in a service.
They form the \glspl{check} of the \gls{examiner}
and will be separated from the other backend logic
The frontend \gls{source-code} has the ability to use these services,
and can call them in any order, use their responses
and supply responses from one service to another.
It also \gls{source-code} displays the user interface to the end user
and contacts the backend for sending and receiving the necessary information.
Persistence of data is managed through some kind of database system
containing the domain.
The backend and the \glspl{check} have access to this domain.

For any future development of this project
the domain can be expanded
and services can be added
without the need to make complex changes to the existing code base.
Only the frontend \gls{source-code} needs to be changed
to make use of the added services.
