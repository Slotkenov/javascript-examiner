% Backend #62
In this chapter the backend of the \gls{examiner} will be revealed in more 
detail. With the structure of the backend already revealed in the Architecture 
chapter, we can move immediately towards the server side logic.

Before listing and explaining the used packages, it makes sense to spend a few 
words on npm\footnote{\url{https://www.npmjs.com/}}. Using Node.js requires
usage of additional packages quite frequently, due to the limited
features contained within the platform itself. This is where npm steps in. npm
is the default package manager for Node.JS, and since version 0.6.3 even bundled
\footnote{\url{http://en.wikipedia.org/wiki/Npm_28software29}} with Node.js.
In order to organize, manage and easily distribute the used packages, it's
common to keep a file named ``package.json'' in the root of the project. All 
relevant information concerning the used packages (like (upper) version number,
whether it's a dependency for the application itself or development only) should 
be kept here, ideally together with information of the application 
itself (like name, repository location,
license)\footnote{\url{https://docs.npmjs.com/files/package.json}}

As just describing the used 
packages\footnote{\url{https://github.com/Slotkenov/javascript-examiner/blob/master/package.json}}
is insufficient when it comes to grasp the way the \gls{examiner} works, the 
packages will be grouped and described from the function they fulfill. In
addition, not every package will be explained, because some are added for client
side purpose --- these are mentioned in the Checks chapter ---, 
or are used for utility purpose --- see maintainablility chapter --- only .

\section{Routing}
An important feature in the application is routing. To realize the intended
flexibility and scalability when it comes to adding \glspl{check} even from
external sources, the routing is an important aspect. To meet this requirement,
Express \footnote{\url{https://docs.npmjs.com/files/package.json}} is used for
this purpose. This choice happened to turn out quite nice. It both made the
realization of a quite moderate webserver very easy, as it provided all
necessary features to create proper RESTful
services.\footnote{\url{http://en.wikipedia.org/wiki/Representational_state_transfer}}

\section{Authentication}
Initially, the scope of the current project was limited to the development of a
prototype. However, in a later stadium the expectations somewhat shifted to a
more production ready version, or at least a testable version in a production
environment. To make this possible, proper authentication would be necessary to
enable user management for both students and tutors. Once again, our choice of 
Node.js proved to be a good one, as we were able to create basic authentication
functionality relatively easy. 

Initially, we implemented Passport\footnote{\url{http://passportjs.org/}} 
with Local Authentication 
Strategy\footnote{\url{https://github.com/jaredhanson/passport-local}}, 
which enabled server side authentication based on users stored in the local
database. The passwords are encrypted before saved to the database, using
bcrypt-nodejs\footnote{\url{https://www.npmjs.com/package/bcrypt-nodejs}}

A nice feature of Passport is the --- nowadays commonly used --- 
possibility to add external authentication mechanisms to the application as 
well, like the ones from Google and Facebook. 
Express-session\footnote{\url{https://github.com/expressjs/session}}
has been added to keep track of sessions. The session IDs are stored in MongoDB using 
connect-mongo\footnote{\url{https://github.com/kcbanner/connect-mongo}}. 

Later, while implementing user management, it became clear there was no plan for
user enrollment --- like password distribution --- for new users. We were able to
implement enrollment functionality based on this
tutorial\footnote{\url{http://sahatyalkabov.com/how-to-implement-password-reset-in-nodejs/}},
using Nodemailer\footnote{\url{http://www.nodemailer.com/}} to send tokens
which enables a password reset.

\section{Persistence}
To enable communication between the Node.js server and MongoDB, the 
Mongoose\footnote{\url{http://mongoosejs.com/}}
package is used. Until now, this chapter did not show much of criticism towards 
the packages of choice, or even about the choice it self. When it comes to the 
decision of chosing Mongoose, it is wise to reflect a moment, especially when 
considering the current implementation.

Exploring the available packages for a given functionality can take some time,
simple because there are a lot packages, and it can be intensive to get a good
idea of the quality and usability for the intended usage. Sometimes it's
tempting to start building functionality for the application based on a package
while in fact still deciding whether the package suites well. In retrospective,
this somewhat happened here. Grasping the working of Mongoose became getting it
to work in the application. As a result, Mongoose is implemented through an 
extra module (``database.js''). This resulted in a lot of additional 
unit-tests, as the usage of the package diverted from the regular approach. 
Furthermore, we were required to use the Models from Mongoose, but did not
refactor earlier and for the same purpose created objects. Finally, after 
recognizing the proper implementation of Mongoose, the adapter is being bypassed
for some database operations (like for saving feedback).

It's not a forgone conclusion whether Mongoose would have been used if the
research had been more thoroughly. However, it's sure there is much to improve
in the current implementation when considering this area. One way is to refactor
the application and replace the initial data Objects with Mongoose models and
getting rid of the adapter. There are alternative packages available too, like 
the core MongoDB driver for Node.js, optionally in combination with a wrapper 
such as mongoskin. 

On the upside, thanks to the surplus of related unit-tests, refactoring might be 
relatively easy, as breaking the application will very likely be reflected in 
the test result.

\section{Future work}
In the last iteration it became clear that the logic concerned with the 
database interaction should be refactored. With regards to other 
functionality, possible improvements might be enabling
authentication with a chosen username (for now using the email address is
mandatory), or even through third parties 
(as mentioned, like Google or Facebook).

