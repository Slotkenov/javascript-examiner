% requirements plusminus 2 pagina's

\section{Requirements for current development}
\subsection{Functional Requirements}
\subsubsection{Interaction}
\begin{description}
  \item[Exercise management] The tutor should be able to create an exercise
    consisting of a description, a solution and one or more unit tests. The
    tutor should also be able to modify and delete existing exercises.
  \item[Overview of exercises] The student should be able to retrieve an
    overview of the available exercises.
  \item[Exercise description retrieval] The student should be able to retrieve
    the description of an exercise.
  \item[Code submission] The student should be able to submit the JavaScript
    code solution he created.
  \item[Feedback] The student should receive feedback on the submitted
    JavaScript code solution.
\end{description}

\subsubsection{Examination}
\begin{description}
  \item[Unit testing] The JavaScript-examiner should be able to execute unit 
    tests for given JavaScript code.
  \item[Code layout definition] The JavaScript-examiner should have a
    definition of a code layout to be used in \glspl{check}.
  \item[Code layout check] The JavaScript-examiner should be able to check if 
    the submitted JavaScript code adheres to a defined code layout.
  \item[Maintainability metrics calculation] The JavaScript-examiner should be 
    able to calculate the cyclomatic complexity, Halstead complexity and the 
    logical lines of code (LLOC) from the submitted JavaScript code.
  \item[Feedback generation] The JavaScript-examiner should generate feedback
    according to the results from the various \glspl{check} and metrics calculations
    performed, and show these to the student.
  \item[Feedback on platform and exercises] Students should be able to provide
    feedback on JavaScript-examiner in general and exercises in particular.
\end{description}

\subsection{Non-functional Requirements}
\begin{description}
  \item[Extensible exercises] The JavaScript-examiner should be designed in a
    way it's possible to extend the scope of exercises and even add support for
    other programming languages in the future. 
\end{description}

\subsubsection{Feedback}
\begin{description}
  \item[Useful] The JavaScript-examiner should provide feedback in a way
    the student is supported to improve his skills.
  \item[Motivational] The JavaScript-examiner should provide feedback in such a
    manner that it motivates the student to continue.
  \item[Elegance] The JavaScript-examiner should return useful and well written 
    feedback, or at least encourage the tutor to write well written feedback.
\end{description}

\section{Requirements for future development}
\subsection{Functional Requirements}
\begin{description}
  \item[Plagiarism detection] The JavaScript-examiner should be able to detect
    plagiarism.
  \item[Relative execution time calculation] The JavaScript-examiner should be 
    able to calculate the relative execution time for given JavaScript code.
  \item[Language constructs detection] The JavaScript-examiner should be able
    to detect, and point out to the student, when an important language
    construct is missing that is advised to be used in the exercise in
    question.
  \item[Student test case submission] The students should be able to write and
    submit test cases for exercises, which the JavaScript-examiner can examine
    and add to the test suite of the particular exercise.
  \item[Dynamic configuration] All aspects of the JavaScript-examiner should be
    dynamically and easily configurable (e.g.\ code layout definitions,
    programming guidelines, metrics comparison margins).
  \item[Feedback among students] The JavaScript-examiner should facilitate 
    communication among the students so they can help each other with the 
    exercises. 
  \item[Store progress] The JavaScript-examiner should help the students to
    keep track of their progress.
  \item[Statistics] The tutor should be able to retrieve statistics about the
    results of the various metrics and \glspl{check} performed in the past on the
    JavaScript code of (any or some of) the students.
  \item[Security in executing submitted code] Sufficient measures should be
    taken to guarantee the integrity of the system while executing submitted
    code.
  \item[Format] The format of the returned feedback should be easily printable.
\end{description}

\subsubsection{Examination}
\begin{description}
  \item[Programming guidelines definition] The JavaScript-examiner should have
    a definition of a set of programming guidelines to be used in \glspl{check}.
  \item[Programming guidelines check] The JavaScript-examiner should be able to
    check if the submitted JavaScript code conforms to a defined set of
    programming guidelines.
  \item[Metric results comparison] The JavaScript-examiner should be able to
    compare the results from the maintainability metrics of the student's
    submitted JavaScript code with the results of the tutor's solution.
  \item[Code saving] The JavaScript-examiner should save the JavaScript code
    from the students, for future examination.
  \item[Results logging] The JavaScript-examiner should log the results from
    the various \glspl{check} and metrics calculations performed, for future
    generation of statistics for the tutor.
  \item[Progress logging] The JavaScript-examiner should log the progress for each
    student (while maintaining anonymity).
\end{description}
