Here a more detailed explanation is given about the feedback-mapper.
This is part of the back-end.
In every \gls{check} that is caled by the \gls{examiner}, a call is made to the feedback-mapper.
The input of the mapper function is the possible feedback of the \gls{check}.
The mapper functions looks for a file that is called check-xxx-feedback,
where xxx is the name of the \gls{check} the mapper is called by.
If the file exists, the content of the file, which is in JSON,
is stored in memory as a key-value pair.
Otherwise, a new JSON object is created.
The mapper function determines which \gls{check} it is called by.
It then decides wich field of the feedback object it takes to match to the key.
This can be the string that is generated by the \gls{check}, or a errorcode. 
If it can't find a matching key, the key-value pair is added to the JSON object,
where the key and value both hold the same item.
If a key is found, the accompanying value is used.


Then the value is stored in the feedback object.
This feedback object is used to give feedback to the client.

The feedback file can be editted to rename or translate the standard feedback at need.
This way the feedback can be altered to give a more descriptive message.

    
\section{Future work}
In the latest version of the program the functionality of the feedback
mapper is limited to only edit the content of the file with an external editor.
In the future,  a more convenient way can be developed to edit the text file.

