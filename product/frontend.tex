% (product 3) Frontend #61
%
% - Polymer
%   - Isolated elements
%   - Core and paper elements
% - Code Mirror

The frontend handles the interaction with the user of the application.
It renders the user interface with data fetched from the backend,
thus facilitating user interaction.
We wanted to make the user experience as much unobtrusive as possible,
in order for the student to be able to put all his attention
to the exercise, submitting his solution and receiving feedback.
Therefore we opted for a \gls{spa}
in which there are no page loads
and all communication with the backend is done through
XHR requests\footnote{\url{https://developer.mozilla.org/en-US/docs/Web/API/XMLHttpRequest?redirectlocale=en-US&redirectslug=DOM\%2FXMLHttpRequest}}.
This makes it possible
to use a relatively simple
RESTful\footnote{\url{https://en.wikipedia.org/wiki/Representational_state_transfer}}
interface on the backend,
but alse means more logic is needed on the frontend.
We choose to use
Polymer\footnote{\url{https://www.polymer-project.org/0.5/}}
as a framework on the frontend
to help us build a solid foundation, and structure or application.

\section{Polymer}
Polymer is a project backed by Google which promotes the use of standards.
The philosophy\footnote{\url{https://www.polymer-project.org/0.5/docs/start/everything.html}}
behind the framework is to make use of existing standards
concerning HTML, \gls{js} and the browser.
Where such a standard does not exist,
it tries to propose a standard which will be useful to the framework
and to other frameworks and developers.
In order to make use of standards
that have not been implemented by browsers yet,
polyfills\footnote{\url{https://www.polymer-project.org/0.5/docs/start/platform.html}}
are provided.
Polyfills are \gls{js} modules
which simulate a certain functionality documented by a standard
not supported within a specific browser yet.

With this philosophy, Polymer is an innovative framework,
while making sure it's compatible with both browsers
and other frameworks.
While it needs polyfills to run on (most of) todays browsers,
as soon as a browser implements a certain standard
Polymer switches from the polyfill to the native code automatically.
In this way there is no need to make any changes or release a new version of 
your frontend code.

\subsection{Web components}
One very interesting standard the Polymer framework builds upon is
web components\footnote{\url{https://www.polymer-project.org/0.5/platform/custom-elements.html}}.
This standard comprises of several aspects;
custom elements, HTML imports, templates and the shadow DOM.
This standard gives you the ability
to extend the standard set of HTML elements with custom elements.
With the HTML import functionality you can load these custom elements
in an convenient manner.
On top of this it's possible to load custom elements created by others as well.
When creating a custom element you can make use of one or more templates,
and behind the scenes custom elements make use of the shadow DOM.

\subsubsection{Isolation}
A great feature of these custom elements is that they live in isolation.
The \gls{js-code} you write within a component is private to that component.
Any CSS styles you write are for that component only,
which means there is no interference with CSS styles of the application
or other components; this way the danger of, in example, name clashes is 
reduced to a minimum.
The same goes for the ID's you define
on HTML elements in a component's template.
The template is private to the component, and so are the ID's you define.
They won't bleed into the final HTML document they will be part of.

These aspects make for a great architecture,
to keep the frontend modularized with self contained components.
And they improve on writing frontend applications
compared to previous frameworks and architectures.

\section{Code Mirror}
Because the \gls{student} needs to enter \gls{js-code}
for his solution to an exercise
and be able to receive feedback on that code
we needed to implement a text editor to display the code
and let the \gls{student} be able to alter the code easily.
We started of with a simple HTML textarea field
which is not very suitable for displaying code though:
It has no syntax highlighting
and tabs have an indentation size of eight spaces.
Furthermore, it would be very difficult to show feedback
next to the line of code it is concerned with.

We thought there must be an editor out there
which would have all the features we needed.
And there was: Code Mirror\footnote{\url{http://codemirror.net/}}.
It has syntax highlighting,
is well suited for \gls{js}
and has automatic indentation.
With code mirror you can also make use of a so called
gutter\footnote{\url{http://codemirror.net/doc/manual.html\#api_decoration}},
a vertical bar next to the editor.
We used the gutter to display a warning icon next to the line of code
on which feedback was received.
When hovering such an icon an information message is displayed,
pointing not only to the line in question,
but also to the very character the feedback message wants to point out
(another feature of Code Mirror).

Code Mirror is also used in other important projects,
like the developer tools in
Google Chrome\footnote{\url{https://developer.chrome.com/devtools}} and
Mozilla Firefox\footnote{\url{https://developer.mozilla.org/en-US/docs/Tools}}.
It is also
well documented\footnote{\url{http://codemirror.net/doc/manual.html}}
and therefore a good choise for the \gls{examiner}.
Successive teams working on the \gls{examiner}
can easily make use of many more of
the functionalities provided by theCode Mirror if needed,
for it is very flexible.
