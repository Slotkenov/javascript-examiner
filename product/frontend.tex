% (product 3) Frontend #61
%
% - Polymer
%   - Isolated elements
%   - Core and paper elements
% - Code Mirror

The frontend handles the interaction with the user of the application.
It rendes the user interface with data fetched from the backend,
and watches for user interaction with that interface.
We wanted to make the user experience as much unobtrusive as possible,
in order for the student to be able to put all his attention
to the exercise, submitting his sulution and receiving feedback.
Therefore we opted for a single page application
in which there are no page loads
and all communication with the backend is done through XHR requests.
This makes it possible
to use a relative simple RESTful interface on the backend,
but alse means more logit is needed on the frontend.
We choose to use Polymer as a framework on the frontend
to help us build a solid foundation and structure or application.

\section{Polymer}
Polymer is a project backed by Google which promotes the use of standards.
The philosophy behind the framework is to make use of existing standards
concerning HTML, \gls{js} and the browser.
Where such a standard does not exist,
try to propose a standard which will be useful to the framework
and to other frameworks and developers.
In order to make use of standards
that have not been implemented by browsers yet,
polyfills are provided.
Polyfills are \gls{js} modules
who simulate a certain functionality documented by a standard,
which is not supported within a specific browser yet.

With this philosophy Polymer is an innovative framework,
while making sure it will be(come) compatible with browsers
and other frameworks.
And while it needs polyfills to run on (most of) todays browsers,
as soon as a browser implements a certain standard
Polymer switches from the polyfill to the native code automatically.
No need to make any changes or release a new version of your frontend code.

\subsection{Web components}
One very interesting standard the Polymer framework builds upon
is web components.
This standard comprises of several aspects;
custom elements, HTML imports, templates and the shadow DOM.
This standard gives you the ability
to extend the standard set of HTML elements with custom elements.
With the HTML import functionality you can load these custom elements
in an easy manner.
And you could load custom elemnts created by others just as well.
When creating a custom element you can make use of one or more templates.
And behind the scenes custom elements make use of the shadow DOM.

\subsubsection{Isolation}
A great feature of these custom elements is that they live in isolation.
The \gls{js-code} you write within a component is private to that component.
Any CSS styles you write are for that component only,
which means there is no interferance with CSS styles of the application
or other components, no name clashes.
The same goes for the ID's you define
on HTML elements in a component's template.
The template is private to the component, and so are the ID's you define.
They won't bleed into the final HTML document they will be part of.

These aspacts make for a great architecture,
to keep the frontend modularized with self contained components.
And they improve on writing frontend applications
compared to previous frameworks and architectures.
