% (product 2) Checks #59
% 
% Here the four checks should be discussed
% - functionality
% - format
% - maintainability
% - syntax


\section{Format}

The format of the JavaScript within the solution is checked with the standard
% Voetnoot link naar JSCS website
JavaScript Code Style (JSCS) code style checker\footnote{\url{http://jscs.info/}}.
It's configuration can be set through a configuration file, jscs-config.json.
% Linken naar de google preset config file
Currently the content of this file is {"preset": "google"} which sets the checker
to the Google preset.
This can be configured at will, depending on the wishes of the tutor.
The Google preset is a reflection of the style guide and is a list of dos and
don'ts for JavaScript
programs\footnote{\url{https://google-styleguide.googlecode.com/svn/trunk/javascriptguide.xml}}.
JavaScript is the main client-side scripting language used by many of Google's
open-source projects.


\section{Functionality}
% Onveiligheid beschrijven van executen (ongeveriefierde code)

This check uses the Mocha framework \footnote{\url{http://mochajs.org/}} to run tests on the submitted code.
To do this, the lines of code that were entered have to be executed.
This could be a threat.
Because the code is taken from the text field on the UI, there is no check
on the lines of code whether these are valid \{js-code}.
This way possible malicious code can be intercepted.


\section{Maintainability}
var esprima = require('esprima');
var escomplex = require('escomplex');
var walker = require('escomplex-ast-moz');
This check returns an Abstract Syntax Tree if that can be produced.
http://en.wikipedia.org/wiki/Abstract_syntax_tree
This is the representation of the submitted code, and can be input to various other tools.

\section{Syntax}
var esprima = require('esprima');
var UglifyJS = require('uglify-js');

//configuration for esprima parsing are included in the check-syntax.js file.


\section{Future work}
An extention of the checks could be the validation of
every single line of code to confirm that it is valid \gls{js-code}.
Also the AST can be used to do more checks and say something about the
equality of the submitted code and the standard solution.