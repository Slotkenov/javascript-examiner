% (product 2) Checks #59
% 
% Here the four checks should be discussed
% - functionality
% - format
% - maintainability
% - syntax

% Algemeen
As it stands, we have 4 checks that are used in the \gls{examiner}.
These check are called from within the server.js file.
The number of checks can easily be extended by adding more endpoints to the list
of allready available \glspl{check}

\section{Format}

The format of the JavaScript within the solution is checked with the standard
% Voetnoot link naar JSCS website
JavaScript Code Style (JSCS) code style checker\footnote{\url{http://jscs.info/}}.
It's configuration can be set through a configuration file, jscs-config.json.
% Linken naar de google preset config file
Currently the content of this file is {"preset": "google"} which sets the checker
to the Google preset.
This can be configured at will, depending on the wishes of the tutor.
The Google preset is a reflection of the style guide and is a list of dos and
don'ts for JavaScript
programs\footnote{\url{https://google-styleguide.googlecode.com/svn/trunk/javascriptguide.xml}}.
JavaScript is the main client-side scripting language used by many of Google's
open-source projects.


\section{Functionality}
% Onveiligheid beschrijven van executen (ongeveriefierde code)

This \gls{check} uses the Mocha framework \footnote{\url{http://mochajs.org/}} to run tests on the submitted code.
To do this, the lines of code that were entered have to be executed.
This could be a threat.
Because the code is taken from the text field on the UI, there is no check
on the lines of code whether these are valid \gls{js-code}.
This way possible malicious code can be intercepted.


\section{Maintainability}

The maintainability \gls{check} uses esprima\footnote{\url{http://esprima.org/}},
escomplex\footnote{\url{https://github.com/philbooth/escomplex}} and
escomplex-ast-moz\footnote{\url{https://github.com/philbooth/escomplex-ast-moz}}.
The first module generates output that is used as input for the next module.
Its output is send to the third and the total output is the software
complexity analysis of the submitted code described in a number of metrics.

\section{Syntax}

Checking the syntax is done by calling the Esprima parser\footnote{\url{http://esprima.org/}}.
This is a light and fast parser, which returns an abstract syntax tree if there
is no error, otherwise it returns the error.
The configuration of the parser can be set within the check.

\section{Future work}
An extention of the \glspl{check} could be the validation of
every single line of code to confirm that it is valid \gls{js-code}.
Also the AST can be used to do more \glspl{check} and say something about the
equality of the submitted code and the standard solution.
