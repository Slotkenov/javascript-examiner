\documentclass{article}

\begin{document}

\title{Dictionary}
\author{Boris Arkenaar}
\date{\today}
\maketitle

\begin{description}
  \item[JavaScript-examiner project] The project in which three students are
    working on software for examining JavaScript code.
  \item[examiner project] See: {\em JavaScript-examiner project}.
  \item[JavaScript-examiner] The software that will be the result of the
    JavaScript-examiner project.
  \item[examiner] See: {\em JavaScript-examiner}.
  \item[JavaScript] The programming language.
  \item[code] Code of any size written in a programming language.
  \item[JavaScript code] Code written in {\em JavaScript}.
  \item[source code] Code written by one of the software developers of the
    JavaScript-examiner project.
  \item[syntax (of {\em JavaScript code})] The order and combination of low
    level {\em language constructs}. {\em JavaScript code} is syntactically
    correct if it adheres to the official definition of the JavaScript language
    ({\em ECMAScript}\footnote{http://www.ecmascript.org}). {\em JavaScript
    code} is either syntactically valid or invalid, this does not depend on the
    assignment.
  \item[functionality (of {\em JavaScript code})] The output {\em JavaScript
    code} should produce on a given input for a specific assignment. This can
    be tested with unit tests. {\em JavaScript code} is either functionally (as
    defined by the assignment) correct or incorrect.
  \item[semantics (of {\em JavaScript code})] The way {\em JavaScript code} is
    constructed of combinations of {\em language constructs}. Can be used to
    determine the quality and efficiency of {\em JavaScript code}.
  \item[language constructs] The smallest meaningful entities with which {\em
    JavaScript code} can be constructed (e.g. a variable definition, a for loop
    or an if statement). Or a combination of such constructs for a specific
    function.
  \item[exercise] An assignment in the {\em JavaScript-examiner} for the {\em
    student} to improve his {\em JavaScript} skills. The student produces
    {\em JavaScript code} for submission to the {\em examiner}.
  \item[solution] {\em JavaScript code} forming a (correct or incorrect)
    solution to an exercise.
  \item[tutor] The person that aims to support the {\em student} in learning
    {\em JavaScript}.
  \item[student] A person that is enrolled to the course {\em Web applications:
    the Client Side}.
  \item[user] A person that visits an {\em online learning platform}.
  \item[Web applications: the Client Side] A course offered by the Open
    University as part of the study (Technical) Computer Science.
  \item[feedback] Communication with the purpose of reflecting on someones 
    learning activity.
  \item[tool] Existing bundle of code specific for a particular
    functionality. Can be used as part of a software product via an API. Can
    also be used stand alone via the command-line. (e.g. JSHint, Acorn.)
  \item[execution environment] An installed software environment for the
    execution of code (e.g. Node.js, JVM).
  \item[framework] A generic code base for building a specific software
    solution (e.g. Workshopper). Can be implemented by configuration and/or by
    modifying and expanding on the code base.
  \item[platform] Specific implementation of a {\em framework}
    (e.g. Nodeschool.IO).
  \item[online learning platform] A {\em platform} which has as goal to give
    people the ability to learn about a subject online.
\end{description}

\end{document}
