\documentclass{article}

\begin{document}

\title{Dictionary}
\author{Boris Arkenaar}
\date{\today}
\maketitle

\begin{description}
  \item[Javascript examiner project] The project in which three students are
    working on software for examining Javascript code.
  \item[examiner project] See: {\em Javascript examiner project}.
  \item[Javascript examiner] The software that will be the result of the
    Javascript examiner project.
  \item[examiner] See: {\em Javascript examiner}.
  \item[Javascript] The programming language.
  \item[Javascript code] Code written in Javascript by a student (for
    submission to the examiner).
  \item[source code] Code written by one of the software developers of the
    Javascript examiner project.
  \item[syntax (of {\em Javascript code})] The order and combination of low
    level {\em language constructs}. {\em Javascript code} is syntactically
    correct if it adheres to the official definition of the Javascript language
    ({\em ECMAScript}\footnote{http://www.ecmascript.org}). {\em Javascript
    code} is either syntactically valid or invalid, this does not depend on the
    assignment.
  \item[functionality (of {\em Javascript code})] The output {\em Javascript
    code} should produce on a given input for a specific assignment. This can
    be tested with unit tests. {\em Javascript code} is either functionally (as
    defined by the assignment) correct or incorrect.
  \item[semantics (of {\em Javascript code})] The way {\em Javascript code} is
    constructed of combinations of {\em language constructs}. Can be used to
    determine the quality and efficiency of {\em Javascript code}.
  \item[language constructs] The smallest meaningful entities with which {\em
    Javascript code} can be constructed (e.g. a variable definition, a for loop
    or an if statement). Or a combination of such constructs for a specific
    function.
\end{description}

\end{document}
