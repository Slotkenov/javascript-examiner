\documentclass{article} 

\usepackage[utf8]{inputenc}
\usepackage{hyperref}
\usepackage[square, numbers]{natbib}

\begin{document} 

\title{Domain research --- Semantics}
\author{Boris Arkenaar}
\date{\today}
\maketitle 

% Analyse van kwaliteitsaspecten van code waarop kan worden gecontroleerd,
% uitzoeken hoe deze kunnen worden uitgevoerd.

\section{Introduction} 

Our goal is to make life easier for a programming language teacher. In our
bachelor project we will develop a
JavaScript-examiner\footnote{\url{https://github.com/Slotkenov/javascript-examiner/}}
which will relieve the teacher from checking all students' assignments and
giving them appropriate feedback in an online course. At he same time, the
JavaScript-examiner will be for the benefit of the students as well. The
teacher simply doesn't have time to check the assignments of all the
students. And therefore the teacher cannot give individual feedback to each
student. The teacher will pick the biggest and most common pitfalls he sees in
the assignments he picks at random. Then he will give the students general
feedback on those pitfalls in the next online group session.

The reason why the JavaScript-examiner will benefit students --- besides
relieving the teacher of too much of a workload --- is threefold.  First off
the JavaScript-examiner is able to check all the assignments of all
students. Secondly, the JavaScript-examiner can give the students immediate
feedback on their assignment --- instead of them having to wait for the next
online group session with the teacher. And finally the JavaScript examiner
gives the students individual feedback. Instead of some general explanation
about common pitfalls the student might not even have had trouble with, the
JavaScript-examiner gives each student feedback on the specific problems
encountered in his assignment.

This article will be an analysis of the quality aspects of JavaScript code on
which the JavaScript-examiner can focus. I will look for existing quality
aspects of code and ways of examining those aspects. Those could be aspects for
programming languages in general, or for a specific programming language. I
will determine how they would apply to JavaScript. And maybe there are quality
aspects specific for JavaScript.

The quality aspects will be compared on various points. I will determine how
good they will apply to JavaScript code, and to code from students' assignments
in particular. They will be compared based on how precise they are in their
feedback. And I will base my comparison on how realistic it would be to
implement inspection of those quality aspects in our project. At the end of this article I will formulate a conclusion based on these comparisons, in the form of an advice on which quality aspects to let the JavaScript-examiner examine.

\section{Code quality}

Code quality can be divided into two areas. In one area you look at the quality
of the code itself, without executing it. You look at aspects like how the code
is laid out, how easy it is to read the code and how well you can understand
what the code is trying to do. These aspects have no effect on the execution of
the code however. The second area looks at the quality of execution. How well
does the code execute? Does it use efficient code structures? In this section we will be looking at the first area: the quality of the code.

\subsection{Code layout}

For readability it is important to structure your code in a way that conveys clearly the intention of the code. There are many different ways you could layout your code and there is not one layout that is better than all the others. Most important is consistency.

The easiest way to check for a consistent layout would be if one particular layout would be required from the students. Then the JavaScript-examiner can simply check for that layout. Otherwise it would have to determine the layout of a particular assignment on various aspects and see if that layout is used consistently throughout the entire assignment.

Besides being consistent in using one layout there are still layout aspects
that would be considered bad practice and should be avoided at all times. There
are many tools for examining code to make sure none of these bad practices are
used in a piece of code. Even specific for JavaScript such tools
exist.\footnote{For instance JSLint (\url{http://www.jslint.com/}) and the less
strict variant JSHint (\url{http://www.jshint.com/})}

\subsection{Maintainability}

Various metrics can be used to give an idea about the maintainability of a
piece of code. We are not looking for measuring maintainability specifically,
but an indication of it can help us determine the quality of the code, for
maintainability is an aspect of good quality code. \citet{rakic2013problems}
list the following metrics used in tools for analyzing code:
\begin{itemize}
  \item Lines of code;
  \item Cyclomatic complexity;
  \item Halstead complexity;
  \item Object oriented metrics.
\end{itemize}
We will look at the first three metrics mentioned. The last one --- Object
oriented metrics --- is not too applicable in our case, for JavaScript is not
as object oriented as, for instance, Java is.

First off you can look at the amount of code in terms of lines of code
(LOC). The more lines of code, the more complex it gets. There are different
ways to determine the LOC though. You can simply count all the lines in the
source files, which we will call lines of code (LOC). But you could also skip
the lines which contain comments and empty lines, which we will call source
lines of code (SLOC). Going one step further you can count the actual
statements in the source code instead of the physical lines. That will give
you the logical lines of code (LLOC). A disadvantage of LLOC is that it is more
difficult to calculate. Instead of simply counting the lines of a text
file. You would need serious knowledge of the programming language in question.

Only looking at the lines of code is not very useful. We can compare the LOC of
the student with the solution of the teacher, but that will give you only a
rough idea about whether the student used far too much code or extraordinarily
few. Next up is cyclomatic complexity. This basically tells you in how many
unique paths your code could be executed \citep{website:js-complexity}. When
the cyclomatic complexity increases, so does the complexity of your code. The
more possible execution paths, the more difficult it is to reason about your
code. The cyclomatic complexity increases when the code base
increases. Therefore it is more useful to express the cyclomatic complexity
relative to the lines of code. This gives us the cyclomatic complexity density
\citep{gill1991cyclomatic}.

The Halstead complexity looks at operators and its operands. It determines the
amount of operators and operands in a piece of code. It also determines the
amount of distinct operators and operands.  Besides that it counts the amount
of distinct operators and operands. With that date some calculations are
performed to determine the complexity of the code. While it is not necessary to
fully understand the programming language for performing this metric, the
Halstead method has some shortcomings as explained by
\citet{yu2010survey}. Where the cyclomatic complexity looks at the control flow
of the code, the Halstead methed only looks at the operations performed, but
ignores the control flow completely.

\section{Execution quality}

The quality can be determined by looking at the code structures that have been
used, and by looking at the manner in which the code structures have been
combined.

There is an interesting way to look at a new programming language which can
also help us here to look at JavaScript code, of what structures it is composed
and how they are combined. These words from Abelson and
Sussman\footnote{Abelson, H., Sussman, G.J. 1984 {\em Structure and
Interpretation of Computer Programs}. MIT Press,
[\url{https://mitpress.mit.edu/sicp/}]} give us that way of thinking: ``What
are the primitives, what are their means of combination, and what are their
means of abstraction?''

To follow these words we would first have to look at the primitives of
JavaScript. Then we can determine in what ways these primitives can be
combined. Finally we look at the means of abstraction JavaScript offers us.

\subsection{Relative execution time}

The relative execution time of a piece of code can be measured
mathematically. As explained by \citet[Chapter 4]{goodrich2008data} the
relative execution time can be expressed using seven functions (constant,
logarithm, linear, n-log~n, etc.). To determine with which function the
relative execution time of a piece of code can be expressed one has to look at
the primitives used in that code. For each primitive the relative execution
time has to be determined. Then you can calculate the relative execution time
of a combination of those functions (i.e. the combination of the primitives
that make op the code).

This method would be a good way to see if the student has used an algorithm
that is to complex for the given assignment. If, for example, the solution of
the teacher executes in n-log~n time while the code of the student executes in
quadratic time, feedback could be given to the student that he can use a more
efficient algorithm for that assignment. The question would be: how to give the
student more concrete feedback? Instead of just telling him he can do better it
would be constructive to point him in the right direction. However, this would
require more knowledge of the particular assignment and the difference between
the solution of the teacher and the solution of the student.

An implementation of this method would require more than determining the
primitives used in a piece of code and how they are combined. When a loop is
used, for instance, you would need information about run time variables to
determine how many times the loop would be iterated over. In other words:
looking at the primitives is not enough, you would need a good understanding of
the algorithm used and the problem that is being solved. This problem might be
solved by requiring the teacher to specify algorithm information when creating
an assignment in the JavaScript-examiner. However further study will be
required to figure out how this would be best implemented, and what information
would be required exactly from the teacher.

\subsection{Code structures}

When looking at the meaning of the code you can distinguish different code
structures. When you look at an assignment, it might be the case that a good
solution should contain one or two specific code structures. That means that we
can look for those structures in a student's code. If they are not there, or if
there are too much of them, the student can be given feedback mentioning this
fact. In order to be able to give the student a more profound explanation of
why his code is not optimal we would need input form the teacher. Because our
goal is to relieve the teacher from too much work we could try and implement
this feature in a similar way \citet[Section 3.2]{watson2011learning} proposed
in their article. They describe a way to provide feedback on how to correct a
piece of code that generates an error. The first time the system encounters a
specific error it has to request the teacher for feedback. The system saves
that feedback into a database along with the error. On any subsequent
occurrence of that same error in any of the students' code the system retrieves
the teacher's feedback from the database. It does not need to bother the
teacher while the student receives valuable feedback.

We would need to determine first what code structures are important. When a
student uses a for loop where the teacher does not, that would be an
interesting case. But when a student uses one more simple assignment statement
than the teacher, that will probably not be very important for instance. We can
determine the set of primitives used in the code, by type and by quantity. And
compare that with the solution of the teacher. This gives us a set of
primitives the student did not use while the teacher did, and of the other way
around. After filtering these two sets of primitives for important ones we have
something to give feedback on. Now we can use the same mechanics as
\citet{watson2011learning} described. We look in the database for the feedback
on a particular primitive that was used --- or not used --- and give that to
the student. In case no feedback was found in the database, the teacher is
requested for input.

Research would be needed to see if this would work the way I describe it
here. Would it be feasible to determine how to rate the primitives of the
JavaScript language? And would the system find all the problem cases or will it
miss some, or find cases that are of no importance? Besides that this functionality would have to be created from the ground up because it is quite specific and I have not found something like it for JavaScript.

\section{Comparing code}

The great thing is: we do not need to determine the quality of a piece of code
on its own. We can compare it to a well defined base piece of code that is
assumed to be correct and efficient (i.e. an acceptable solution). That means
that we do not need to decide how many lines of code would be too much, or what
cyclomatic complexity would make the code too complex. We can just compare the
lines of code, and the halstead complexity of the student's code with the
solution of the teacher. We do have to determine however: how bad is the code
if it is not as good as the solution of the teacher? There should probably be
some margin of acceptance.

The code layout is an absolute check to see if the student's code adheres to a
well defined set of rules of best practices in coding. The code either fails on
some points or it is completely laid out according to the layout definitions
used. How you go about giving feedback when some rules are broken is another
question. The JavaScript-examiner could say the code is incorrect or more
gently tell the student that it could be better. It could clearly point out
where the shortcomings are or simply tell the student to do better. The point
is that the rules of code layout can be well defined, how strict you hold the
student to it is a matter of feedback.

A different matter are the other metrics for code and execution quality. They
do not simply give you an answer whether the code is good or not. It is more of
a gray area and you would need to determine when code is good, when it is
acceptable and when it would need improvement. These acceptability margins will
be different for each assignment (e.g. a more complex assignment should allow a
higher cyclomatic complexity). The teacher could be asked to specify these
margins when creating an assignment, but the teacher might struggle to
determine good margins for the assignment at hand. Another way might be to ask
the teacher to write a solution to the exercise himself. The
JavaScript-examiner could then determine a baseline for the different metrics
by calculating the metrics for the teacher's solution. That would give the
system a good base for which it can say: when I get these results (or better
results) for the metrics a on a piece of code, the code is good, otherwise the
code could be better. Preferably you would still want some form of margin,
where the results of the metrics slightly below those of the teacher's are as
acceptable as the teacher's. Determining how those margins can best be
calculated is subject to further research.

\section{Conclusion}

While no one metric is perfect it would be best to combine several metrics to
determine the quality of the code and of its execution. Looking at code layout
is a good starting point and will be easy to implement because many libraries
exist for performing these checks for JavaScript code. The other metrics will
require a baseline which can be obtained by analyzing the teacher's
solution. This won't be o problem to implement as well but would require
further research and development for determining the right margins of
acceptance of these metrics.

The maintainability metrics are well defined and have been implemented in
different libraries. Therefore these are also a good candidate to implement in
the JavaScript-examiner. Some extra research might be necessary though to find
the right implementation, use and combination of these metrics for best
representing the quality of the students' code.

Looking at execution quality of the code would be an interesting area to look
into, but will take more time and effort to implement. To calculate the
relative execution time good understanding of the code and the algorithm are
needed. Also, because of its complexity, no ready to be used libraries exist
for executing this specific metric. Serious research and development will be
needed to be able to implement it.

Building up a database with specific usages of primitives for an assignment
along with helpful feedback for a student seems an interesting solution. The
JavaScript-examiner will be able to generate valuable feedback, simply and
effectively by asking the teacher. At the same time the teacher will be
relieved from too much work because the system would save that feedback into a
database. Further research is needed to determine the best way of extracting
the important primitives from code. Development is needed to implement this
functionality and make sure the feedback from the database will be supplied at
the right times.

\bibliographystyle{abbrvnat}
\bibliography{bibliography}

\end{document}
