\chapter{Conclusion}

Commencing a project is complex. Only a few decisions are yet taken, while
the direction and scope is still vague and undefined. This section reflects in a 
glance on the journey that led to the genesis of the \gls{examiner}, pointing 
out the main aspects and key decisions. Apart from the \gls{examiner}, the from 
research derived conclusions and recommendations are gathered here as well. 
While the thesis up to now has been presented from a more or less chronological
narrative, this conclusion will start with the current state (which can be best
experienced through actually 
examining \footnote{installation instruction can be found at
\url{https://github.com/Slotkenov/javascript-examiner\#installing}}
the \gls{examiner}, but for a visual glance,
please checkout the user manual ingredients appendix \ref{app:user-manual}). From
here, shuttling back and forth in time, it will denote and elucidate this state,
shaping perspective and context.

When considering this current state, a first indicator is the \gls{examiner}
itself. While writing this section, the product owner is actually in power to
test the application in a production environment. When considering the
initial goal was creating a prototype, this achievement deserves some attention.
During the research and elaboration phases, the main focus was directed towards 
the functionalities in examining the \gls{js-code} and producing \gls{feedback}
as a result of this examination. In demonstrating the application at the
first iteration, this started to change. The product owner hinted the desire of
a more or less production ready version at the end of the project. The 
importance of enabling an early release of the \gls{examiner} came across
during the Research Context related consult as well. Whereas a main goal of the
studied project (IDEAS) was supporting the \gls{tutor}, there were no actual
results available to determine whether this was achieved. According to the
consulted stakeholder, they did not manage to test the application in a 
production environment yet. This endorsed the importance of the desire to 
develop towards a production release early on.

The initial approach for developing the \gls{examiner} targeted at a flexible
and highly maintainable architecture and implementation, as it is not clear
what the eventual usage will be. This requirement was immediately stressed
when shifting from prototyping to production orientation. It turned out as
expected, as we where able to add functionality like user management without 
altering the structure of the application.

As we speak, it is actually possible to deploy the \gls{examiner} and use it in 
the \gls{wac} course. The extent to which this actual benefits to achieve 
the main goals
--- reduce the labor intensity while maintaining or improving feedback to 
students --- can only be determined afterwards. Still, the results from 
\gls{tutor} and \gls{student} usage probably will be of great guidance when 
determining the way to go from here. (This immediately gives rise to the need of
creating a way for presenting the user feedback. It is already possible to
address feedback on the \gls{examiner} and individual \glspl{exercise},
but this user feedback is only accessible through querying the database.)
Besides the deducible indicator through feedback from users, this 
thesis issues quite some recommendations. Research revealed
features which could not be developed within the given time. Striking
examples are conclusions derived from the domain researches, like the possibility
to enforce the usage of a design recipe like ``Felleisen''. The research itself
led to new research questions as well. Whereas the paper on writing elegant
\gls{js-code} elucidated the importance of more research on semantics, the
consult directed towards helpful research that seems to be a fruitful approach
to actually deal with this semantics related complexity.

Developing software is a constant deliberation between refactoring and adding
functionality. While the logic can be deduced over and over again, a pragmatic
approach is demanded to reach functional goals. The Product part describes these
considerations, leaving suggestions for improvement too. The possibilities to
extend the program, mainly arisen from research, be at odds with the urge of
refactoring the current code. In particular, the security issue in the 
functionality check should be dealt with, possibly by using one of the offered
resolutions. Another addressed concern, that is the database adapter, should be
refactored, as the difficulty and time intensity to do so will grow when extending
the \gls{examiner} with features that depend on persistence.

The recommendations addressed is this section are only a subset. A more 
comprehensive overview can be found on GitHub. As already mentioned in the 
introduction, both the \gls{source-code}, 
installation instructions, user manual (wiki) as the recommendations and known 
bugs (issues) are published with 
GitHub\footnote{\url{https://github.com/Slotkenov/javascript-examiner}}. 
The clustering of relevant data and assignment
of issues has been very helpful to keep track of the progress. Another benefit 
is the ease of transmission to a new team. Not only are they able to ``fork'' 
the current version of the code, it is possible to check the list of 
remaining open issues as well.

Elaborating this project has been challenging in a positive way. In the beginning
it took some time to get the process going, in the end it was hard to stop developing
new functionality, because of the growing enthusiasm about the \gls{examiner}. 
It is our believe the \gls{examiner} has the potential to really contribute in
the initial goals. The student will already be able to benefit from the feedback,
whereas the tutor has to invest in adding exercises, which is likely to result in
fewer examination requests from students. Thus, a first step towards automatic 
\gls{js-code} examination has been taken.
