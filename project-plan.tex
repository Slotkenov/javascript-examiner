\documentclass{article}

\begin{document}

\title{Project plan}
\author{Ronald Kluft \and Bram Nieuwenhuize \and Boris Arkenaar}
\date{\today}
\maketitle

\section{Introduction}
%\\*
%\\*Geef een korte toelichting op de achtergronden van het projectplan:
%\\*Wat ging eraan vooraf?
%\\*Welke stappen zijn genomen?

%\\*Geef vervolgens een toelichting op de opbouw van het projectplan als leeswijzer. Per hoofdstuk geeft u in één of twee zinnen %\\*aan wat er in dat betreffende hoofdstuk staat. Tot slot geeft u in de inleiding een voorstel tot uitvoering aan.
%\\*Dat wil zeggen:
%\\*Wanneer denkt men te starten met de uitvoering van het project?
%\\*Indien er tijd zit tussen de afronding van het projectplan en de start van het project, wat wordt er in de tussentijd gedaan?

In chapter 2 we explain that background of this project is to be found in the search for a way to improve Javascript education. We see a good example of how Javascript is being educated at the OU.

3 states that the objective of the project is to create a Program to automatically check programcode constructed of Javascript and HTML.
Where the ultimate goal is to provide automatic feedback to Javascript code that uses an User Interface

In 4 we explain that for realizing this project we will start off with the individual domain research to find out in what direction it would be best to proceed with the project.

Chapter 5 explains the structure of the project and the stakeholders.

\section{Initial Situation}
%\\*Bram
%\\*Beantwoord de vragen:
%\\*Waarom wil de opdrachtgever dit project?
%\\*Wat is de huidige situatie?

The background of this project is to be found in the search for a way to improve Javascript education, whereby distance education is the dominant form. This form should be kept in mind, for it changes the set of educational instruments, being held against a traditional form like classroom education. For example, the amount of interaction between teacher and student and students among each other is no major aspect of distance education. This complicates the possibility to provide accurate feedback. It may be argued this problem transcends distance education, as being a general problem of mass education in general. This might be true, and only strengthens the importance of this project, accounting that receiving feedback is a crucial aspect of aquiring certain skills and knowledge. 

Learning a programming language like Javascript is not a straightforward process. Programming in general has many facets, there are many different styles and ways to perform this art. Provided with some basic techniques, a few syntax rules and a method for running a written program, it's quite easy for a student to get some feedback, even on the particular written elaborations of exercises: the written code executes (with (un)expected result), or it produces some kind of error. Is this feedback useful? 

\subsection{Feedback in the current situation}
%\\*Welke problemen en oorzaken zijn de aanleiding geweest tot de wens om te veranderen?
%\\*Hoe ziet de omgeving eruit? Beschrijf deze vanuit het oogpunt van de opdrachtgever.

The course 'Webapplications: The Client Side'\footnote{http://www.ou.nl/studieaanbod/T58221} (developed and distributed by the Open University <Voetnoot>) is a good example of how Javascript is being educated at the Open Univerity. A student is being provided with some theoretical material, and a set of exercises and corresponing 'proper' solutions. In the current situation, students elaborate a given exercise. To check whether the result is adequate, they can validate it with the provided answer, and/or submit it for review by the professor. Both options have some shortcomings: in programming there are often multiple good solutions for a problem, so it’s not unlikely the elaborated answer is fundamentally different from the provided one. In these cases the provided answer has no value in terms of giving feedback. Submitting the answer for review overcomes this problem, but leads to a time and labor-intensive task for the professor. In order to keep it manageable, the professor scans the submitted answers to select common errors and difficulties, and then provide some feedback on this selection. This requires students to translate this generalization back to their own answer, just like what happens in comparison an specific answer with the provided answer.

Returning to the question whether the feedback gained by executing code is valuable. The description of the current situation does not explicitly answers the question, though the alternative methods to generate feedback suggests that the execution of code is not sufficient. Solving a problem by writing a javascriptprogram is a creative process, with often a multitude of more or less correct solutions. Because of this variation in solutions, running the code does not suffice to ensure a student has written a proper solution. The currently used alternative techniques (provide a proper (one of the often many) solution or manual reviewing by a professor) both have some serious flaws. 

Another subject should be taken in mind as well. The Open University has adopted an educational model that goes by the name of Activation Education. An important aspect in this model is learning by doing, where the student is encouraged to take the freedom of using the own creativity and decisiveness in order to solve a problem. This stresses the use of 'proper' solutions, in the sense working towards that specific solution will not contribute to the intended education style. Reviewing solutions for problems that encourages creativity and decisiveness are likely harder to interpret and compare in a search for common errors, resulting in an even bigger workload for the professor. 

\subsection{Initial Documentation}
%\\*Welke documentatie ligt ten grondslag aan het project?
%\\*Welke kwaliteiten heeft deze documentatie?
%\\*Welke activiteiten moeten er nog verricht worden om deze documentatie te complementeren?

Harry Passier, intiator and primary stakeholder of this project, provided us with some guidance in the form of literature, academic papers and working groups. These might be of use during this project. A significant part of the first parts of the next phase will be used to explore the contents of these artefacts:
\begin{itemize}
  \item Parts of PHD initiator, namely: Chapter 1,2 until page 36, figure 2.2(conditions of possibility to provide feedback) ,Epiloque (classification), Page 139).
  \item Research Group ‘Vakdidactiek Informatica (Passier, Stuurman, Pootjes)), especially to get a model for exploring the semantic characteristics of producing solutions in Javascript. )
  \item Webapplications: the Client Side course material
\end{itemize}

\section{Projectresult}
%\\* Ronald
%\\*
%\\*Geef antwoord op de vraag:
%\\*Wat is het uieindelijke resultaat van het project?

\subsection{Objective}
%\\*Wat is het achterliggende doel van de opdrachtgever?
%\\*Wat is de koppeling tussen bedrijfsprocessen en de omgeving, gezien door de bril van de opdrachtgever?

The objective of the project is to create a Program to automatically check programcode constructed of Javascript and HTML. This to reduce the amount of time spent by the Professors to correct the solutions to exercises and giving useful and constructive feedback to enhance value to the student. 
This will result in the Professor having more time to teach. 

\subsection{Result}
%\\*Wat is aan het einde van het project het concrete resultaat?

The ultimate goal is to provide automatic feedback to Javascript code that uses an User Interface, but the first step is to give automatic feedback to stand-alone Javascript.

\subsection{Quality}
%\\*Wat zijn de kwaliteitseisen die gesteld worden aan het eindresultaat?

The quality requirements of the result are of the least priority. There should be at least a possibility to upload solutions to the questions, and there should be given a feedback to the student. The most important thing is that there is a working program. If there is time, the user input and given feedback can be extended and enhanced.

\subsection{Assignment}
%\\*Wat is de projectopdracht in precieze bewoordingen?
%\\*Wat wordt wel en wat wordt niet gerekend tot de opdracht?
%\\*Wat zijn de eisen en beperkingen die de opdrachtgever stelt aan tijd, geld, mensen of middelen?
%\\*Wie is de opdrachtgever en wat zal hij bijdragen aan het realiseren van de opdracht?

The projectassignment : "Realiseer een Program voor het automatisch nakijken van Javascript en HTML."

The included tasks are :
\begin{itemize}
  \item Analyse a usercontext
  \item Architecture of a (technical) solution
  \item Designing a Program
  \item Building a Program
  \item Implementing a Program
  \item Literature study
\end{itemize}

There is a time constraint. The amount of time that can be spent on the project is 8 months by 3 students, where each students spends about 400 hours.

The projectowner is Dr. ir. Harrie Passier\\
He will contribute by giving feedback, asking constructive questions, make go/no-go decissions and checking the work allready done.

\subsection{Riskfactors}
%\\*Welke risico's zijn er bij het behalen van het gewenste resultaat?
%\\*Welke zijn het belangrijkste en hebben de hoogste prioriteit, en welke een lagere?
%\\*Welke maatregelen worden er genomen?

There are several risks that can be:
\begin{itemize}
  \item A student quits
  \item There is no literature
  \item There is no platform to implement the Program
  \item There is not enough time to build a Program
  \item There is no feedback from the project owner
  \item There are several versions of documentation
  \item There is no communication between students
\end{itemize}

\noindent To handle these risks:
\begin{itemize}
  \item Expand the domain
  \item Let the projectowner supply a platform
  \item Start early with  a very small version and expand this along the way
  \item Alert the mentor
  \item Use a subversion or GIT system
  \item Alert the mentor
\end{itemize}

\subsection{Successfactors}
%\\*Wat zijn de succesfactoren die de opdrachtgever onderkend heeft?

The project will be a success if there is a working Program on wich other students can expand, without having it to build it from scratch again.

\section{Project phases}

%\\*Boris
%\\*Dit hoofdstuk biedt een antwoord op de vragen:
%\\*Op welke manier wordt het projectresultaat gerealiseerd?
%\\*In welke fasen wordt het projectresultaat gerealiseerd?

For realizing this project we will start with the individual domain research to find out in what direction it would be best to proceed with the project. We will then be able to expand this project plan in more detail. The next phase will be the development of the required software divided in multiple milestones. Alongside the development another research will take place in the form of a consult with a researcher from an area on which we might contribute to one another.

\subsection{Introduction}

%Welke aandachtsgebieden en welke fasen worden gebruikt?\\

Focus areas:
\begin{itemize}
  \item Checking javascript syntactically and functionally
  \item Checking javascript semantically
  \item Making use of existing code
  \item Building a system for managing assignments
  \item Building a front-end for submitting code
\end{itemize}

Phases:
\begin{itemize}
  \item Domain research
  \item Consult researcher from an active research
  \item Development
\end{itemize}

%Hoe ziet de resultatenmatrix eruit?\\

\subsection{Domain research}
%\\*Welke activiteiten in welke fasen zijn hier nodig?
%\\*Wat zijn de tussenresultaten die deze activiteiten opleveren?

\subsubsection{Feedback and didactical aspects}
To get insight in possible ways to create a program that produces useful feedback and adds value to the didactical aspects of Javascript education, a study will be performed from an pedagogical aspect. Several educational platforms with a aim to programming in general or Javascript in particular will be studied. These approaches will be held against relevant studies about feedback and didactics (the PhD of the initiator might be a good use here).
\subsubsection{Checking javascript semantically}
What are the fundamental semantical entities that can exist within javascript code? When we have such a set, can we programatically determine of what semantic entities a given piece of Javascript code consists. How do we define a correct semantic solution, and compare a given solution with it? And in a more broad sense, are there scientific models available to review programming language from a semantical point of view.
\subsubsection{Technical aspects of similar tools and possibilites for reuse}
Find out what related tools, libraries and software already exist in this area and in what ways they could be re-used for this project. Starting with a general overview of the aspects (programming language source code, licencing, interfaces etc.) of these software units, the most interesting ones can be selected and more closely evaluated to determine whether it's possible to use, combine and or distribute them. The results from this study provides information in which programming language(s) and architecture should be used for the realization of the program.

\subsection{Consult researcher from an active research}
%\\*Welke activiteiten in welke fase zijn hier nodig?
%\\*Wat zijn de tussenresultaten die deze activiteiten opleveren?
We will have to determine which researcher we want to consult with and on what kind of research. After we have done the domain research and have expanded the planning further we can also think about the details of this consult. Then we can specify on what focus areas this consult has some influence. For example, if the technical domain (for this particular kind of program) raises important questions, it might be useful to consult someone with knowlegde in this particular area, like the 'IDEAS vakgroep'.\footnote{http://ideas.cs.uu.nl/}\\

\subsection{Development}
%\\*Welke activiteiten in welke fase zijn hier nodig?
%\\*Wat zijn de tussenresultaten die deze activiteiten opleveren?
We can plan this phase in more detail, and separate it into multiple phases, after we have determined a more detailed direction. This will be the case after we have done our domain research.

\subsubsection{Checking javascript syntactically and functionally}
This will be our main focus for the first working version of the software.
\subsubsection{Checking javascript semantically}
Checking for semantics is far more advanced and will therefore only be done if there will be enough time to complete this. Also the domain research might shed some light on the possibilities of this area.
\subsubsection{Making use of existing code}
We will include libraries into our software where the domain research has pointed out that would benefit the progress of this project. If the domain research directs us to a complete software package to be used, we will deploy a local working version of that package to start with. After which we will expand on the software.
\subsubsection{Building a system for managing assignments}
Further into the development process we will build an interface into which a teacher has the ability to define assignments.
\subsubsection{Building a front-end for submitting code}
After the first working version we will also develop a front-end at which a student can submit an exercise and receive feedback from the software about the javascript code.
\subsection{Dependencies}
%\\*Zijn er tussen de verschillende activiteiten relevante afhankelijkheden en zo ja, welke?
The domain research is an important first part of this project. It gives us insight into the further direction of this project and is therefore mandatory for a successful completion of the other activities.

\section{Projectframe}
%\\*In dit hoofdstuk geeft u antwoord op de vragen 'wie', 'waarmee' en 'waar'.
%\\*Binnen welk kader mag het project zich begeven om het eindresultaat te realiseren?

\subsection{Introduction}
%\\*Wat is het belang dat dit onderwerp, het projectkader, heeft voor het welslagen van het totale project?
%\\*Wat zijn de voorwaarden die de projectmanager stelt, zowel naar het project (intern) als naar de omgeving (extern)?

If there is a good foundation, which will be built in this project, the possibilitie for extension are endless. It will lead to better education of futrure software engineers.  

\subsection{Projectorganisation}
%\\*Wat zijn de profielschetsen?
%\\*Wat is het organogram en/of het verantwoordelijkheidsschema?

\begin{itemize}
  \item Harrie Passier - projectowner
  \item Bram Nieuwenhuize - projectmember
  \item Boris Arkenaar - projectmember
  \item Ronald Kluft - projectmember
  \item Annemiek Herrewijn-VdZande - mentor
\end{itemize}

Responsibilities:
\begin{itemize}
  \item Bram: 
	\begin{itemize}
	\item Domain research: Feedback and didactical aspects of currently available Javascript examination tools. (chapter 4.2.1)
	\end{itemize}
  \item Boris:
	\begin{itemize}
	\item Domain research: Semantics and Javascript: oppertunities and bounderies for automatic semantical examination and feedback (chapter 4.2.2)
	\end{itemize}
  \item Ronald:
	\begin{itemize}
	\item Domain research: Technical aspects and reuse options of currently available Javascript check tools and api's. (chapter 4.2.3)
	\end{itemize}
\end{itemize}

We do not have a project manager or specific roles within the project, but rotate tasks in an adhoc and pragmatic way.

\subsection{Conditions to project owner}
%\\*Wat zijn de voorwaarden die de opdrachtgever dient te realiseren?

No current preliminaries

\subsection{Conditions to third parties}
%\\*Wat zijn de eisen die gesteld worden aan derden?

Access to documentation/literature\\
Access to implementation platform

\subsection{Projectcommunication}
%\\*Welke formele communicatie is van belang voor het slagen van het project, zowel intern als extern?

To achieve the project goal the formal communication on the milestones and a constructive feedback are essential.

\subsection{Facilities and support}
%\\*Welke faciliteiten en hulpmiddelen zijn er nodig om het project te realiseren?

Subversion or GIT server.

\subsection{Procedures en guidelines}
Welke procedures en richtlijnen binnen en buiten het project zijn nodig om het project te laten slagen?

\section{Projectplanning}
%\\*
%\\*Dit hoofdstuk geeft antwoord op de vragen:
%\\*Wanneer wordt het eindresultaat opgeleverd?
%\\*Wanneer worden de tussenresultaten opgeleverd?
%\\*Welke mensen en middelen zijn wanneer nodig?
%\\*Wat zijn de financiële consequenties?
Our planning is based on the predescribed workload for the project (400 hours for each projectmember: 1200 hours total), and a weekly investment of 15 hours a week by each person. To make a realistic planning, at first we will try to adhere to the given workload for each phase.  Though, in order to maintain a pragmatic approach, there are some predefined moments for reconsidering the planning, in case the project requires us to make alternations in the planning. The next phases, Domain Research and the Consult, will clarify the specific activities for the remainder of the project. Therefore, these phases are planned in a specific manner, while the subsequent phases (Development and Finishing essay) are more loosely defined until the contents have become clear. Thus, as with many aspects of this project, the planning develops iteratively (a Rational Unified Process approach). Still, the deadlines of the milestone-products are fixed, to ensure completion within the scheduled time.

\subsection{Norms and Assumptions}
%\\*Welke normen, bijvoorbeeld wat betreft het ziektepercentage, worden gehanteerd bij de planning?
As recommended, we will initially assume an effectiveness of 80\%. This means there is a 20\% margin for fall outs like sickness, personal circumstances etc. . Week 52 of 2014 and the first week of 2015 are not scheduled (holidays). 

\subsection{6.2 Activities}
%\\*Hoeveel tijd kost het om de activiteiten, zoals ze beschreven staan bij de projectfasering, uit te voeren?
%\\*Wat zijn de afhankelijkheden tussen de activiteiten?
\begin{tabular}{| p{3cm} | c | p{3.5 cm} | p{3.5 cm} | r | }
  \hline
  Phases: & Scheduled: & Activities: & Milestone document: & Deadline \\ \hline
  Domain \& Techniques & sept - oct & 1. General Research \newline 2. Individual Research \newline 3. Writing papers \newline 4. Discuss results  & Individual Paper & 03-11-2014  \\ \hline
  Research context & oct - dec & 1. Research \newline 2. Consult plan \newline 3. Consult \newline 4. Write paper & Paper &  15-12-2014 \\ \hline
  Design \& Implementation & nov-mar & Rational Unified Process artefacts & Working Beta for each iteration &  01-04-2015\\ \hline
  Documentation & nov-mar & 1. Developing Thesis \newline 2. Documentation not covered by RUP & Documentation up to date at deadline of each iteration &  01-04-2015\\ \hline
  Finishing Thesis \& Presentation & apr & 1. Examination thesis \newline 2. Presentation artefacts \newline 3. Presentation & Final Thesis & 01-05-2015 \\ \hline
  \hline
 \end{tabular}

\subsection{Iterative approach}
The parallel phases 'Design \& Implementation' and 'Documentation' will be completed iteratively, according to the Design and Implementation phases of the Rational Unified Process (and, depending on the result (Program or product) testing and deployment as well). The preliminary schedule for each iteration is as follows (the specific activities will depend on decissions to made during the next phase, i.e. building a Program or product):

\begin{tabular}{ c | c || c  }
  \hline			
  Iteration & Period & Deadline \\ \hline \hline
  1 & nov-dec & 15-12-2014 \\ \hline
  2 & dec-jan & 26-01-2014 \\ \hline
  3 & jan-feb & 26-02-2014 \\ \hline
  4 & feb-mar & 26-03-2014 \\ \hline
  \hline  
\end{tabular}

At the end of each iteration there should be a functioning beta version, whereby each iteration results in added functionality/features.

\subsection{Evaluate project planning }
In order to keep the progress manageable, the following planning activities are scheduled, to ensure enduring focus and facilitate possible alterations to the planning. Missing specifics can be added as well during these activities. These activities will be added to the agenda of the regular meetings:

\begin{tabular}{ l | p{3 cm} }
  \hline			
  Activity & Scheduled \\ \hline \hline
  Specify iteration activities (see 6.3) & 03-11-2014 \\ \hline
  Scheduling contact moments stakeholders &  27-10-2014 \newline 17-11-2014 \newline 15-12-2014 \newline 26-01-2015 \\ \hline
  Evaluate progress towards final product & 01-12-2014 \newline 12-01-2015 \\ \hline
  Evaluate meeting schedule for the remainder of the project &  19-01-2015 \\ \hline
  \hline  
\end{tabular}

\end{document}
