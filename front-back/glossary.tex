\newglossaryentry{project}
{
  name=JavaScript--examiner project,
  description={The project in which three students are working
    on software for examining JavaScript code.}
}
\newglossaryentry{examiner}
{
  name=JavaScript--examiner,
  description={The software that will be the result of the \gls{project}
    \footnote{\url{https://github.com/Slotkenov/javascript-examiner/}}.}
}
\newglossaryentry{js}
{
  name=JavaScript,
  description={The programming language.}
}
\newglossaryentry{code}
{
  name=code,
  description={Code of any size written in a programming language.}
}
\newglossaryentry{js-code}
{
  name=JavaScript code,
  description={Code written in \gls{js}.}
}
\newglossaryentry{source-code}
{
  name=source code,
  description={Code written by one of the software developers
    of the \gls{project}.}
}
\newglossaryentry{syntax}
{
  name=syntax (of \gls{js-code}),
  description={The order and combination of low level \glspl{construct}.
    \gls{js-code} is syntactically correct if it adheres
    to the official definition of the \gls{js} language
    ({\em ECMAScript}\footnote{http://www.ecmascript.org}).
    \gls{js-code} is either syntactically valid or invalid,
    this does not depend on the assignment.}
}
\newglossaryentry{functionality}
{
  name=functionality (of \gls{js-code}),
  description={The output \gls{js-code} should produce
    on a given input for a specific assignment.
    This can be tested with unit tests.
    \Gls{js-code} is either functionally (as defined by the assignment)
    correct or incorrect.}
}
\newglossaryentry{semantics}
{
  name=semantics (of \gls{js-code}),
  description={The way \gls{js-code} is constructed
    of combinations of \glspl{construct}.
    Can be used to determine the quality and efficiency of \gls{js-code}.}
}
\newglossaryentry{construct}
{
  name=language construct,
  description={The smallest meaningful entity
    with which \gls{js-code} can be constructed
    (e.g.\ a variable definition, a for loop or an if statement).
    Or a combination of such \glspl{construct} for a specific function.}
}
\newglossaryentry{exercise}
{
  name=exercise,
  description={An assignment in the \gls{examiner} for the \gls{student}
    to improve his \gls{js} skills.
    The student produces \gls{js-code} for submission to the {\em examiner}.}
}
\newglossaryentry{solution}
{
  name=solution,
  description={\Gls{js-code} forming a
    (correct or incorrect) solution to an exercise.}
}
\newglossaryentry{tutor}
{
  name=tutor,
  description={The person that aims to support the \gls{student}
    in learning \gls{js}.}
}
\newglossaryentry{student}
{
  name=student,
  description={A person that is enrolled to the course \gls{wac}.}
}
\newglossaryentry{user}
{
  name=user,
  description={A person that visits an \gls{olp}.}
}
\newglossaryentry{wac}
{
  name=Web applications: the Client Side,
  description={A course offered by the Open University
    as part of the study (Technical) Computer Science.}
}
\newglossaryentry{feedback}
{
  name=feedback,
  description={Communication with the purpose of
    reflecting on someones learning activity.}
}
\newglossaryentry{tool}
{
  name=tool,
  description={Existing bundle of \gls{code}
    specific for a particular functionality.
    Can be used as part of a software product via an API.
    Can also be used stand alone via the command-line.
    (e.g.\ JSHint, Acorn.)}
}
\newglossaryentry{environment}
{
  name=execution environment,
  description={An installed software environment
    for the execution of code (e.g.\ Node.js, JVM).}
}
\newglossaryentry{framework}
{
  name=framework,
  description={A generic code base for building
    a specific software solution (e.g.\ Workshopper).
    Can be implemented by configuration and/or
    by modifying and expanding on the code base.}
}
\newglossaryentry{platform}
{
  name=platform,
  description={Specific implementation of a \gls{framework}
    (e.g.\ Nodeschool.IO).}
}
\newglossaryentry{olp}
{
  name=online learning platform,
  description={A \gls{platform} which has as goal
    to give people the ability to learn about a subject online.}
}
\newglossaryentry{code-quality}
{
  name=code quality,
  description={The quality of code in terms of \gls{code-layout}
    and \gls{maintainability}}
}
\newglossaryentry{code-layout}
{
  name=code layout,
  description={Describes the order in which sections of \gls{code} are placed,
    relative to each other,
    in order to keep a good overview of the \gls{code}.}
}
\newglossaryentry{code-formatting}
{
  name=code formatting,
  description={Describes different ways
    to position the \glspl{construct} in a text file,
    and the use of white space, in a given \gls{code},
    without affecting its functionality.}
}
\newglossaryentry{guidelines}
{
  name=programming guidelines,
  description={Defines a set of rules for writing {js-code}
    (e.g.\ redundancy of code, proper constructors, variable declaration.).}
}
\newglossaryentry{maintainability}
{
  name=maintainability,
  description={Describes a given \gls{code} in terms of complexity.}
}
\newglossaryentry{exe-quality}
{
  name=execution quality,
  description={Describes how well a given \gls{code} executes
    in terms of speed and efficient memory usage.}
}
\newglossaryentry{rel-exe-time}
{
  name=relative execution time,
  description={An abstract way of expressing the execution time
    of a given \gls{code} in terms of mathematical functions.}
}
\newglossaryentry{haskell}
{
  name=Haskell,
  description={A functional programming language.
  \footnote{\url{https://www.haskell.org/}}}
}
\newglossaryentry{spv}
{
  name=Semantic Preserving Variations,
  description={An exhaustive set of normalization techniques for \gls{code}.}
}
\newglossaryentry{ast}
{
  name=abstract syntax tree,
  description={A representation of \gls{code} in the form of a tree.}
}
\newglossaryentry{domainreasoner}
{
  name=domain reasoner,
  description={A reasoner for a specific domain. In particular an implementation
    of the IDEAS framework}
}
\newglossaryentry{spa}
{
  name=single page application,
  description={A complex web application with multiple views,
    but without the need to reload
    the page.\footnote{\url{http://www.johnpapa.net/pageinspa/}}}
}
\newglossaryentry{check}
{
  name=examination check,
  description={A service the \gls{examiner} can use
    to examine \gls{js-code} on a specific property,
    like \gls{syntax}, \gls{functionality}, \gls{code-quality},
    \gls{code-layout}, \gls{code-formatting}, \gls{guidelines},
    \gls{maintainability}, \gls{exe-quality}
    and \gls{rel-exe-time}}
}
