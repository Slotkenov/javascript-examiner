% een korte samenvatting (plusminus 250 woorden)

\begin{abstract}
As learning JavaScript is not easy, educating this skill is often difficult and
time consuming too. The Open University offers a course which educates in writing
JavaScript code. Tutors of this course emerged a demand from students, for
receiving proper feedback on the actual JavaScript-code they had written.
As manual examination is too labor intensive, a search for alternatives began.
This search has led to this project, which aims at reducing the workload for
tutors at the one hand, while maintaining or even improving the quality of
feedback on JavaScript-code written by students on the other hand.
On top of these merely pragmatic goals,
the inquiry toward didactics in educating computer science is a
point of interest as well. This thesis generally describes
the outcome of the project.

To get insight in, and grasp the sphere and scope of this and subsequent
projects, the following topics have been researched: Feedback, Semantics in code,
writing code by using a recipe (Felleisen) and Semantic Preserving Variations.
This research led to requirements based on which an application has been
developed. This application has become an easily extendible, modular tool
that is able to automatically provide feedback on, in JavaScript written
solutions to exercises defined by the tutors. The current version supports
checking on syntax, code style, functionality and maintainability.

As both the architecture of the application and the scope of the research
largely orientated beyond the scope of this project,
many ideas for expansion have been uttered within this thesis,
and there are no constraints to expand or deploy the
application in upcoming (research) projects.
\end{abstract}
