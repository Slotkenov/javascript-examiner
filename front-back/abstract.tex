% een korte samenvatting (plusminus 250 woorden)

\begin{abstract}
As learning JavaScript is not easy, educating this skill is often difficult and
time consuming. The Open University offers a course which educates in writing 
JavaScript code. Tutors of this course emerged a demand from students, for 
receiving proper feedback on the actual JavaScript-code they had written. 
As manual examination is too labour intensive, a search for alternatives began.
This search has led to this project, which aims at reducing the workload for 
tutors at the one hand, while maintaining or even improving the quality of 
feedback on JavaScript code written by students.
This thesis mainly describes the result of this project, 
which aims at creating a tool that is able to automatically provide feedback on
solutions (of exercises common to the ones from the current course) that are 
written in JavaScript. To get insight in, and grasp the sphere and scope of this 
project, there has been research towards Feedback, Semantics in code, Felleisen.
This research led to requirements on which we based a design model. With this
design model we are developing a tool that can provide feedback on: Syntax, 
Style, (Functionality, Semantics <<Upcoming iterations>>).
\end{abstract}